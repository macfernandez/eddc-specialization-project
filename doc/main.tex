\documentclass[colorinlistoftodos]{article}
\usepackage[utf8]{inputenc}
\usepackage{natbib}
\usepackage[spanish]{babel}
\usepackage[numbib,nottoc]{tocbibind}
\usepackage{graphicx}
\usepackage{longtable}
\usepackage[hidelinks]{hyperref}
\usepackage{todonotes}
\renewcommand\spanishtablename{Tabla}


\title{
    {\large Universidad de Buenos Aires\vspace{20pt}}\\
    {\includegraphics[scale=0.4]{images/uba.eps}}\\
    {\large\textbf{Facultad de Ciencias Exactas y Naturales}}\vspace{16pt}\\
    {\small\textbf{Especialización en Exploración de Datos\\y Descubrimiento del Conocimiento}\vspace{16pt}}\\
    {\includegraphics[scale=0.5]{images/master.eps}}\\
    {Reglamentación del acceso al aborto:\\selección de rasgos para un análisis discursivo}
}
\author{Autora:\\Fernández Urquiza, Macarena}
\date{Fecha:\todo{Definir fecha}}

\begin{document}
\clearpage\maketitle
\thispagestyle{empty}

\newpage
\tableofcontents

\newpage

\begin{abstract}
\todo[color=blue!40]{Escribir abstract}Abstract.
\end{abstract}

\section{Introducción}

\subsection{Antecedentes}

% Temas a abordar
%% Impacto de la reglamentación del acceso al aborto
%% Importancia del acceso, por parte de los ciudadanos, a los datos
%%  de incumbencia gubernamental y de organización social
%% Posibilidad de hacer un análisis discursivo con mayor cantidad
%%  de datos --> FerCa
%% Relevancia de la buena selección de features para el análisis
%%  de texto y, en este caso, del discurso
%% Paper seleccionado

\subsection{Objetivos}

% objetivos grales: podrían ser los mismos que los
% del paper (?)
% objetivos particulares: podría ser contrastar si
% las observaciones realizada en el paper se
% verifican para el español

% GRAL
% Realizar una selección de rasgos que permita
% capturar las particularidades discursivas de los
% partidos a favor de la despenalización del aborto
% y de los partidos en contra

% PARTICULAR
% - Cotejar si las observaciones detalladas en el paper
% se evidencian también para el caso del español
% - Los rasgos seleccionados permiten entrenar un
% modelo que, dado un discurso, distinga a qué
% tipo de partido pertenece su emisor

\section{Datos}
\subsection{Preprocesamiento}
Para este trabajo se utilizó la versión taquigráfica de la sesión número
23 (reunión 28) celebrada por la Cámara de Senadores, en la que se abordó la
regulación del acceso a la interrupción voluntaria del embarazo y a sus
posteriores cuidados. Los datos fueron obtenidos de la página del Senado de la Nación
Argentina\footnote{\url{https://www.senado.gob.ar/parlamentario/sesiones/29-12-2020/28/}}
utilizando el método de \textit{scrapping}.\par
En primer lugar, se descargó la transcripción en formato \textit{PDF} y, luego, por
medio de la librería \textit{pdfminer}\footnote{\url{https://pdfminersix.readthedocs.io/en/latest/}},
se lo convirtió a texto plano, de manera que fuese procesable. Dado que la librería
empleada en la conversión no arrojó una transcripción limpia y que, además, no todo el texto
resultaba de relevancia para el presente trabajo, se realizó una tarea de
preprocesamiento, con el fin de limpiar y organizar los datos de forma útil a los
objetivos aquí perseguidos.\par
Como primer paso del preprocesamiento se quitaron los encabezados y pie de paǵina.
El archivo \textit{.pdf}, convertido a \textit{.txt}, no contaba con separación de
páginas, por lo que los encabezados y pies se convertían en líneas de texto que
interrumpían los discursos de los senadores y participantes del debate.
Posteriormente, se extrajo la sección del texto pertinente para este trabajo:
la sección 6, dedicada a la ``Regulación  del  acceso  a  la  interrupción
voluntaria  del  embarazo  y  a  la atención postaborto''. Otras secciones consistían
en el izamineto de la bandera, la convocatoria a la sesión, la lectura de la ley
resultante, entre otras cuestiones que no hacían al discurso argumentativos de los
asistentes, sino más bien a cuestiones protocolares, por lo que fueron desestimadas
para este análisis.\par
Con el texto de interés delimitado, y mediante el uso de patrones regulares, se
identificó a los distintos oradores y sus respectivos discursos, como así también
a las secciones que no pertenecían a fragmentos discursivos emitidos durante la
discusión, sino a comentarios agregados por el taquígrafo\footnote{``Luego de unos
instantes'' y ``Contenido no inteligible'' constituyen ejemplos de estos
fragmentos.}. Siguiendo la metodología descripta por \cite{monroe2008fightin},
los datos fueron guardados en un archivo en formato \textit{XML}, en el que
cada fragmento fue etiquetado con las siguiente información:
\begin{itemize}
    \item \texttt{speech}: etiqueta que indica si el fragmento corresponde a un
    discurso efectivamente emitido durante la discusión, en cuyo caso toma el
    valor \texttt{true}, o si consiste en un comentario del taquígrafo, ante lo cual
    toma el valor \texttt{false}.
    \item \texttt{speaker}: en el caso de que el contenido del texto refiera a
    un fragmento discursivo, esta etiqueta indica el nombre del orador que lo
    pronunció. Si el fragmento es un comentario del traquígrafo, la etiqueta
    recibe el valor \texttt{none}.
\end{itemize}
\par
Adicionalmente, como resultado del preprocesamiento, se extrajo la lista de
asistentes a la sesión en formato tabular, con el objetivo de poder
enriquecerla adicionando especificaciónes como la filiación partidaria y la
decisión de voto. En el caso de la filiación partidaria, se utilizó la
librería \textit{selenium}\footnote{\url{https://selenium-python.readthedocs.io/index.html}}
para buscar, en la página del Senado destinada a tal fin\footnote{\url{https://www.senado.gob.ar/senadores/Historico/Fecha}},
los senadores en ejercicio en la fecha en la que tuvo lugar la sesión.
La decisón de voto, en cambio, debió ser transcripta de forma manual. Si bien
se encontraba detallada en el documento de la sesión descarcargado, no fue posible convertir
la tabla del documento \textit{.pdf} a texto plano sin pérdida de información.

\subsection{Descripción}
La sesión analizada contó con un total de 70 senadores,
pertenecientes a 26 partidos políticos, distribuidos del
siguiente modo (Figura \ref{fig-distrib-senators}):

%\begin{figure}[h!]
%\centering
%\includegraphics[scale=0.5]{../visualizations/senators_distribution.png}
%\caption{Distribución de senadores en partidos políticos y en provincias.}
%\label{fig-distrib-senators}
%\end{figure}

Respecto de los partidos, solo uno (Frente de todos) cuenta con 15
senadores; también uno solo (Alianza frente para la victoria) es
representado por 10 senadores; dor partidos (Alianza cambiemos y
Juntos por el cambio) cuentan con 7 senadores, y el resto de los
partidos tienen entre 3, 2 y un senador. Y en cuanto a las provincias
(24 en total), la mayoría tiene 3 senadores exceptuando a Tucumán y
a La Rioja, que tienen solo 2.\todo[color=pink]{Agregar tablas en anexo
dependiendo de la extensión del trabajo.}

La figura \ref{fig-distrib-senators} nos muestra además que la intención
de voto no guarda una relación unívoca con los partidos a los cuales los
senadores representan. A excepción de los partidos que cuentan con un único
senador, la mayoría cuenta con senadores que votaron a favor y senadores
que votaron en contra de la ley para el acceso al aborto.

%\begin{figure}[h!]
%\centering
%\includegraphics[scale=0.5]{../visualizations/senators_vote.png}
%\caption{Distribución votos en los distintos partidos políticos.}
%\label{fig-distrib-vote}
%\end{figure}

Por último, al considerar las intervenciones discursivas, observamos que
9 senadores no intervinieron en la sesión, y las intervenciones de los
restantes 61 senadores fueron resumidas en la Tabla \ref{table-tokens}, que plasma
las métricas univariadas de la cantidad de tokens totales y únicos y nos
permite observar que ambas obedecen una distribución sesgada a izquierda.

%\begin{figure}[h!]
%\centering
%\includegraphics[scale=0.6]{../visualizations/tokens_vote.png}
%\caption{Distribución de tokens emitidos por los oradores en relación a
%su intención de voto.}
%\label{fig-distrib-tokens-vote}
%\end{figure}

La Figura \ref{fig-distrib-tokens-vote}, por otro lado, nos muestra,
de forma más detallada, la distribución de los tokens emitidos según
la intención de voto de los oradores.

\begin{table}[h!]
\begin{center}
\begin{tabular}{ |c|c|c|c| }
\hline
Tokens & Media & Mediana & Desvío Estándar \\
\hline\hline
Totales & 811.65 & 636.12 & 714.79 \\
\hline
Únicos & 296.47 & 254.25 & 238.68 \\
\hline
\end{tabular}
\caption{M\'etricas univariadas de la cantidad de tokens}
\label{table-tokens}
\end{center}
\end{table}


\section{Metodología}
\subsection{Métodos estadísticos}
\subsection{Clasificación}
\subsection{Métodos predictivos}
% Describir métodos aplicados.
%% Model-based ???

%%% NaiveBayes (baseline)
%%% Reg. Logística

%       | RF | DFA | DFR |
% NB    |                |
% RL    |    |     |     |

\section{Resultados}

\section{Discusión y conclusiones}

\bibliographystyle{linquiry3}
\bibliography{bibliography}

\listoftodos%

\end{document}
