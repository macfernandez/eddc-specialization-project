\documentclass[colorinlistoftodos]{article}
\usepackage[utf8]{inputenc}
\usepackage{natbib}
\usepackage{usebib}
\bibinput{bibliography}
\usepackage[spanish]{babel}
\usepackage[numbib,nottoc]{tocbibind}
\usepackage[table]{xcolor}
\usepackage{makecell}
\usepackage{graphicx}
\usepackage{adjustbox}
\usepackage{amsmath}
\usepackage{amssymb}
\usepackage[font=footnotesize]{caption}
\usepackage{longtable}
\usepackage{placeins}
\usepackage[hidelinks]{hyperref}
\setcounter{tocdepth}{2}
\renewcommand\spanishtablename{Tabla}

\definecolor{highlight-blue}{rgb}{
    0.12156862745098039, 0.4666666666666667, 0.7058823529411765
}
\definecolor{highlight-orange}{rgb}{
    1.0, 0.4980392156862745, 0.054901960784313725
}

\hypersetup{
    colorlinks=true,
    urlcolor=blue,
    linkcolor=blue,
    filecolor=black,
    linkcolor=black,
    citecolor=black
}


\title{
    {\large Universidad de Buenos Aires\vspace{20pt}}\\
    {\includegraphics[scale=0.4]{images/uba.eps}}\\
    {\large\textbf{Facultad de Ciencias Exactas y Naturales}}\vspace{16pt}\\
    {\small\textbf{Especializaci\'on en Exploraci\'on de Datos\\y Descubrimiento del Conocimiento}\vspace{16pt}}\\
    {\includegraphics[scale=0.5]{images/master.eps}}\\
    {
        Selecci\'on de t\'ecnicas estad\'isticas\\
        para la vectorizaci\'on de discursos pol\'iticos\\
        referidos a la reglamentaci\'on del acceso al aborto
    }
}
\author{Autora:\\Fern\'andez Urquiza, Macarena}
\date{Fecha: 28 -- 10 -- 2024}

\begin{document}
\clearpage\maketitle
\thispagestyle{empty}

\newpage
\tableofcontents

\newpage

\begin{abstract}
En este trabajo se aborda la selecci\'on y evaluaci\'on de rasgos para el
entrenamiento de modelos de aprendizaje automatizado aplicado sobre datos textuales.
El \textit{corpus} con el que se trabaja pertenece a la transcripci\'on taquigr\'afica
de la sesi\'on del Senado de la Naci\'on Argentina para la reglamentaci\'on del acceso
al aborto, la cual fue obtenida mediante la t\'ecnica de \textit{scraping}. Durante el
procesamiento de los datos se recurre a una lematizaci\'on automatizada sobre la que
luego se realiza una correcci\'on manual. Para la vectorizaci\'on se emplean
m\'etodos estad\'isticos previamente probados entre los que se encuentran la diferencia
de frecuencias absolutas, el \textit{ratio} de \textit{log-odds} y el cálculo de
\textit{TF-IDF}, entre otros. Tras evaluar los distintos vectorizadores respecto de
su rendimiento para el entrenamiento de un modelo de regresi\'on log\'istica,
se utiliza el mejor vectorizador hallado para el entrenamiento final.
\end{abstract}

\newpage

\section{Introducci\'on}\label{section-intro}

\subsection{Antecedentes}\label{subsection-intro-background}
El ``Encuentro nacional de mujeres'' consiste en un evento de alcance nacional
en el territorio argentino que se realiza de manera anual desde 1986 y se propone
auto-convocado, democr\'atico, pluralista y federal.
Como fruto de este evento surge, en 2005, la ``Campaña nacional por el derecho al
aborto legal, seguro y gratuito'', una alianza de organizaciones que promueve y
lucha por la educaci\'on sexual como pilar fundamental en la decisi\'on sobre la gestaci\'on,
el acceso a m\'etodos anticonceptions que permitan la adecuada prevenci\'on de embarazos
no deseados, y el aborto legal, seguro y grauito que le posibilite a la persona gestante
discontinuar con un embarazo si no desea atravesarlo.
Hacia 2007, la Campaña redacta y presenta el ``Proyecto de ley de interrupci\'on
voluntaria del embarazo'' como iniciativa social para que sea considerado por los
legisladores y, as\'i, llegue a ser tratado en el Congreso de la Naci\'on.
Pero no es sino hasta junio de 2018, y luego de reiteradas presentaciones, que logran
contar con el aval de entre 20 y 70 diputados. Reci\'en entonces el proyecto llega a
ser debatido en la C\'amara de dichos legisladores, donde logra media sanci\'on.
Posteriormente, la C\'amara de Senadores lo rechaza.
En diciembre de 2020, el proyecto se vuelve a tratar y, esta vez, logra la aprobaci\'on
del Congreso.
La ley 27.610, sobre el ``Acceso a la interrupci\'on voluntaria del embarazo'', entr\'o
en vigencia en la Argentina el 24 de enero de 2021 y regula el acceso a la
interrupci\'on voluntaria del embarazo y a los cuidados postaborto para todas
las personas con capacidad de gestar.\footnote{\citeauthor{campana@lalucha}.}
\footnote{\citeauthor{huesped@historia}.}
\par
Por otra parte, la ley 27.275 garantiza el acceso a la informaci\'on p\'ublica y permite su b\'usqueda,
acceso y solicitud, como as\'i tambi\'en su an\'alisis, procesamiento, uso y distribuci\'on.
Se considera informaci\'on p\'ublica a todos aquellos datos generados, obtenidos o
controlados por los organismos del Estado y empresas indicados en la misma ley, entre
los cuales se encuentra el Poder Legislativo de la
Naci\'on\footnote{\citeauthor{minjusticia@accesoinfo}}, conformado por la C\'amara
de Diputados y Senadores.
Luego de sus sesiones, ambas c\'amaras disponibilizan en formato digital la
transcripci\'on taquigr\'afica de la reuni\'on, de modo que cualquier ciudadano con
acceso a una computadora e internet pueda acceder a ella.
\par
Al trabajar con textos, el cient\'ifico de datos se ve en la necesidad
de vectorizar, mediante alg\'un mecanismo, el \textit{corpus} que desea
utilizar.
De este modo, lo convierte en una secuencia de n\'umeros capaz de ser
procesada por una computadora siguiendo cierto procedimiento.
Los textos, una vez transformados en vectores, pueden ser agrupados, 
analizados para la obtenci\'on de informaci\'on o utilizados
como conjunto de entrenamiento y testeo de modelos predictivos.
En el cap\'itulo 6 de \usebibentry{jurafsky2000speech}{title}, Jurafsky
repasa los distintos usos y ventajas de la vectorizaci\'on de documentos, a la vez
que detalla los procedimientos m\'as utilizados en aprendizaje autom\'atico. Entre ellos,
menciona el c\'alculo de \textit{TF-IDF} como uno de los m\'etodos \textit{baselines}.
\par
Por su parte, \cite{monroe2008fightin} exponen una serie de t\'ecnicas
que podr\'ian ser \'utiles a la hora de capturar palabras con contenido partidario
en el discurso pol\'itico y los aplican sobre datos tomados del Senado de los Estados
Unidos. Sus objectivos se centran en la selecci\'on y evaluaci\'on de rasgos, a trav\'es
de los cuales esperan distinguir qu\'e palabras indican la adopci\'on de una u otra postura
y, de ah\'i, que puedan usarse como \textit{features} confiables a la hora de entrenar un
modelo. A su vez, respecto de la evaluaci\'on, pretenden no solo comprender qu\'e
palabras caracterizan a determinado partido sino, adem\'as, en qu\'e medida o grado
lo caractirzan, cu\'ales son las palabras m\'as partidarias o representativas de ciertas
posturas.


\subsection{Objetivos}\label{subsection-intro-objectives}
En este trabajo y a partir del acceso a los datos de la C\'amara de Senadores,
se retoman las t\'ecnicas estad\'isticas detalladas por
\citeauthor{monroe2008fightin} y se las utiliza para vectorizar los
discursos emitidos por los legisldores en la vig\'esimo tercera sesi\'on,
en la cual se discuti\'o la ley para el acceso a
la interrupci\'on voluntaria del embarazo.
Se intenta, con esto, evaluar alternativas posibles al tradicional \textit{TF-IDF}
y comparar su desempeño en la selecci\'on de rasgos y entrenamiento de modelos.

Como objetivos generales, este trabajo persigue:

\begin{itemize}
    \item{Comparar distintos m\'etodos estad\'isticos que permitan representar
    discursos en t\'erminos num\'ericos.}
    \item{Seleccionar uno o m\'as m\'etodos estad\'isticos que posibiliten
    generar una representaci\'on vectorial de los discursos a favor y en
    contra de la ley de la interrupci\'on voluntaria del embarazo y entrenar
    con ella un modelo predictivo de clasificaci\'on.}
\end{itemize}

Para lograr dichos objetivos, este trabajo se propone, en t\'erminos
particulares:

\begin{itemize}
    \item{Obtener un \textit{corpus} de datos con discursos a favor y en contra
    de la legislaci\'on del aborto sobre el cual se puedan aplicar las t\'ecnicas de
    vectorizaci\'on y, luego, entrenar un modelo de predicci\'on.}
    \item{En los casos necesarios, implementar los m\'etodos estad\'isticos detallados
    por \cite{monroe2008fightin} en un c\'odigo ejecutable de \textit{Python} que
    luego pueda ser aplicado al \textit{corpus} de discursos.}
    \item{Comparar el rendimiento de los distintos m\'etodos de vectorizaci\'on utilizando
    t\'ecnicas propias del aprendizaje autom\'atico, tales como la validaci\'on
    cruzada, y m\'etricas igualmente acordes, como el \textit{accuracy}, la precisi\'on,
    la cobertura y el \textit{score F1}.}
    \item{Entrenar un modelo predictivo de clasificaci\'on, como la regresi\'on log\'istica,
    y evaluar su rendimiento al utilizar el m\'etodo de vectoriazi\'on seleccionado.}
\end{itemize}

A continuaci\'on, el lector podr\'a encontrar, en el apartado \ref{section-data},
una descripci\'on sobre la descarga y obtenci\'on de los datos utilizados en este
trabajo, como as\'i tambi\'en de las caracter\'isticas generales que presenta. Las
secciones \ref{section-methods} y \ref{section-results} exponen, respectivamente,
los m\'etodos empleados y los resultados obtenidos a partir de su implementaci\'on.
Ambas poseen la misma estructura: en primer lugar hacen referencia
al proceso de anotaci\'on autom\'atico con supervisi\'on y correcci\'on manual utilizado
para pre-procesar el \textit{corpus}; luego presentan la selecci\'on de rasgos
para vectorizar los discursos y lo referido a la comparaci\'on de estos m\'etodos, y
finalmente se detienen en el entrenamiento de un modelo de clasificaci\'on (la
regresi\'on log\'istica) y su correspondiente evaluaci\'on. La secci\'on
\ref{section-discussion} concluye con algunas reflexiones finales y
discusi\'on al respecto. Hacia el final de este trabajo es posible encontrar el
Anexo, donde se detallan las pautas de anotaci\'on adoptadas y tablas y gr\'aficos
adicionales.


\subsection{Estructura del trabajo}\label{subsection-intro-structure}
A continuaci\'on, el lector podr\'a encontrar, en el apartado \ref{section-data},
una descripci\'on sobre la descarga y obtenci\'on de los datos utilizados en este
trabajo, como as\'i tambi\'en de las caracter\'isticas generales que presenta. Las
secciones \ref{section-methods} y \ref{section-results} exponen, respectivamente,
los m\'etodos empleados y los resultados obtenidos a partir de su implementaci\'on.
Ambas poseen la misma estructura: en primer lugar hacen referencia
al proceso de anotaci\'on autom\'atico con supervisi\'on y correcci\'on manual utilizado
para pre-procesar el \textit{corpus}; luego presentan la selecci\'on de rasgos
para vectorizar los discursos y lo referido a la comparaci\'on de estos m\'etodos, y
finalmente se detienen en el entrenamiento de un modelo de clasificaci\'on (la
regresi\'on log\'istica) y su correspondiente evaluaci\'on. La secci\'on
\ref{section-discussion} concluye con algunas reflexiones finales y
discusi\'on al respecto. Hacia el final de este trabajo es posible encontrar el
Anexo, donde se detallan las pautas de anotaci\'on adoptadas y tablas y gr\'aficos
adicionales.



\section{Datos}\label{section-data}

\subsection{Descarga y preprocesamiento}\label{subsection-data-preprocessing}
Para este trabajo se utilizó la versión {taquigr\'afica} de la sesión número
23 (reunión 28) celebrada por la {C\'amara} de Senadores, en la que se abordó la
regulación del acceso a la interrupción voluntaria del embarazo y a sus
posteriores cuidados. Los datos fueron obtenidos de la {p\'agina} del Senado de la Nación
Argentina\footnote{\url{https://www.senado.gob.ar/parlamentario/sesiones/29-12-2020/28/downloadTac}}
utilizando el método de \textit{scraping}.\par
En primer lugar, se descargó la transcripción en formato \textit{PDF} y, luego, por
medio de la librería \textit{pdfminer}\footnote{\url{https://pdfminersix.readthedocs.io/en/latest/}},
se lo convirtió a texto plano, de manera que fuese procesable. Dado que la librería
empleada en la conversión no arrojó una transcripción limpia y que, {adem\'as}, no todo el texto
resultaba de relevancia para el presente trabajo, se {realiz\'o} una tarea de
preprocesamiento, con el fin de limpiar y organizar los datos de forma útil a los
objetivos aquí perseguidos.\par
Como primer paso del preprocesamiento se quitaron los encabezados y pies de paǵina.
El archivo \textit{.pdf}, convertido a \textit{.txt}, no contaba con separación de
{p\'aginas}, por lo que los encabezados y pies se convertían en líneas de texto que
interrumpían los discursos de los senadores y participantes del debate.
Posteriormente, se extrajo la sección del texto pertinente para este trabajo:
la sección 6, dedicada a la ``Regulación  del  acceso  a  la  interrupción
voluntaria  del  embarazo  y  a  la atención postaborto''. Otras secciones consistían
en el izamineto de la bandera, la convocatoria a la sesión, la lectura de la ley
resultante, entre otras cuestiones que no hacían al discurso argumentativos de los
asistentes, sino {m\'as} bien a cuestiones protocolares, por lo que fueron desestimadas
para este {an\'alisis}.\par
Con el texto de interés delimitado, y mediante el uso de patrones regulares, se
identificó a los distintos oradores y sus respectivos discursos, como así también
a las secciones que no pertenecían a fragmentos discursivos emitidos durante la
discusión, sino a comentarios agregados por el taquígrafo\footnote{``Luego de unos
instantes'' y ``Contenido no inteligible'' constituyen ejemplos de estos
fragmentos.}. Siguiendo la metodología descripta por \cite{monroe2008fightin},
los datos fueron guardados en un archivo en formato \textit{.xml}, en el que
cada fragmento fue etiquetado con las siguiente información:
\begin{itemize}
    \item \texttt{speech}: etiqueta que indica si el fragmento corresponde a un
    discurso efectivamente emitido durante la discusión, en cuyo caso toma el
    valor \texttt{true}, o si consiste en un comentario del taquígrafo, ante lo cual
    toma el valor \texttt{false}.
    \item \texttt{speaker}: en el caso de que el contenido del texto refiera a
    un fragmento discursivo, esta etiqueta indica el nombre del orador que lo
    pronunció. Si el fragmento es un comentario del traquígrafo, la etiqueta
    recibe el valor \texttt{none}.
\end{itemize}
\par
Adicionalmente, como resultado del preprocesamiento, se extrajo la lista de
asistentes a la sesión en formato tabular, con el objetivo de poder
enriquecerla adicionando especificaciónes como la filiación partidaria y la
decisión de voto. En el caso de la filiación partidaria, se utilizó la
librería \textit{selenium}\footnote{\url{https://selenium-python.readthedocs.io/index.html}}
para buscar, en la {p\'agina} del Senado destinada a tal
fin\footnote{\url{https://www.senado.gob.ar/senadores/Historico/Fecha}},
los senadores en ejercicio en la fecha en la que tuvo lugar la sesión.
Esta librería permite un interacción dinámica con la página \textit{web} consultada,
por lo que posibilitó el acceso al motor de búsqueda de la del Senado, la selección
del período deseado de forma automática y el procesamiento de los datos de interés,
los cuales se encontraban en \textit{.html} en la página pero que, luego de su
extracci\'on fueron guardados en formato \textit{.csv}.
\par
La decisón de voto, en cambio, debió ser transcripta de forma manual. Si bien
se encontraba detallada en el documento de la sesión descarcargado, no fue posible
convertir la tabla del documento \textit{.pdf} a texto plano sin pérdida de información.
Por lo cual, tomando los nombres de los senadores previamente extra\'idos, se gener\'o
un \textit{csv} en el cual se le agregó a cada uno el voto emitido.
\par
Una vez obtenidos los distintos conjuntos de datos para el estudio en cuestión,
fue necesario un paso ulterior de procesamiento a fin de que la información
recopilada de diversas fuentes pudiera ser vinculada de manera
satisfactoria. El desaf\'io aqu\'i residi\'o en la discrepancia entre
dichas fuentes para nombrar a los senadores. Al incio de la versión taquigráfica
es posible encontrar una lista de todos los asistentes a la sesión del Senado,
incluyendo no solo a los senadores con poder de voto sino también a quienes presidieron
y oficiaron en la secretaría. Esta lista, utilizada para la confección
de la tabla que vincula a cada senador con el voto emitido, enuncia a cada funcionario
recurriendo primero al nombre y luego, al apellido (en los caoss en los que
la persona tiene más de un nombre y/o apellido, se hizo uso de todos ellos). Esta
enunciación, sin embargo, es modificada a lo largo de la transcripción, donde la
primera vez que cada interlocutor hace uso de la palabra, se lo refiere con apellido
y nombre (en ese orden) y luego, solo con el apellido (uno solo en los casos que no
se prestan a confusión, o más de uno en caso contrario). Así también, utilizando primero
el apellido y luego, el nombre es como se hace referencia a los senadores en la página
destinada a proporcionar su lugar de origen y filiación partidaria. Para lidiar con esta
diversidad en la enunciación, se realizó un proceso de estandarización híbrido en el
cual, tomando como base los nombres y apellidos en un \textit{dataset}, se exploró
exhaustivamente con cuáles podían corresponderse en otro \textit{dataset}. Para esto, se
consideró cada cadena de nombre/s y apellido/s como un conjunto y se evaluó si existía
coincidencia entre ambos o si alguno era subconjunto de otro. En la mayoría de los casos,
este mecanismo permitió la desambiguación. Los pocos casos que no pudieron ser estandarizados
de esta manera por darse relación entre más de dos conjuntos fueron desambiguados manualmente.

\subsection{Descripci\'on}\label{subsection-data-description}
La sesión analizada contó con un total de 70 senadores,
pertenecientes a 27 partidos políticos, distribuidos del
siguiente modo (figura \ref{fig-distrib-senators}):

\begin{figure}[h!]
\centering
\includegraphics[scale=0.7]{../visualizations/distrib_histplot_senators_parties.png}
\caption{Distribución de senadores en partidos políticos.}
\label{fig-distrib-senators}
\end{figure}

Solo uno de los partidos (Frente de todos) cuenta con 12
senadores; también uno solo (Alianza frente para la victoria) es
representado por 10 senadores; dos partidos (Alianza cambiemos y
Juntos por el cambio) cuentan con 7 senadores, y el resto de los
partidos tienen entre 3, 2 y un senador.
En cuanto a las provincias (24 en total), todas tienen 3 senadores
exceptuando a Tucumán y a La Rioja, que tienen solo 2.

La figura \ref{fig-distrib-vote} nos muestra además que la intención
de voto no guarda una relación unívoca con los partidos a los cuales los
senadores representan. A excepción de los partidos que cuentan con un único
senador, la mayoría es representado con senadores que votaron a favor y senadores
que votaron en contra de la ley para el acceso al aborto. Aquí también
es posible ver que un senador del Frente Justicialista se abstuvo de votar,
mientras que dos senadores, uno de Cambiemos Fuerza Cívica Riojana y uno de
Frente Unidad Justicialista San Luis, estuvieron ausentes en la votación.

\begin{figure}[h!]
\centering
\includegraphics[scale=0.48]{../visualizations/senators_vote_by_party.png}
\caption{Distribución de votos en los distintos partidos políticos.}
\label{fig-distrib-vote}
\end{figure}

Respecto de las intervenciones discursivas, observamos que, en promedio, se emitieron
2.87 discursos por senador, con un desvío estándar de 6.23 y una mediana de 1, coincidente además
con la moda, lo que nos indicaría que se trata de una distribución asimétrica a derecha.
Estas medidas de centralidad incluyen a 9 senadores que no intervinieron en
la sesión, 7 de ellos votaron en contra de la despenalización del aborto; uno, a favor,
y uno estuvo ausente durante la votación. La figura \ref{fig-distrib-speech} muestra esta
distribución global y también discriminada por intención de voto.
Al desagregar los datos según elección en la votación, vemos que las abstenciones no presentan
dispersión y su media se ubica en el valor 1. Esto se debe a que solo un senador
se abstuvo de votar y, a su vez, solo emitió un discurso.
Una situación similar se da en las ausencias, donde también se observa un único discurso.
Pero dado que dos senadores estuvieron ausentes, la media se ubica hacia el valor 0.5 y
el desvío, hacia el 0.7.
En cuanto a los votos positivos y negativos, ambos presentan una moda de 1, pero los negativos
exhiben una media y un desvío estándar mayor que los positivos: $3.03\pm8.43$ \textit{versus} 
$2.92\pm4.26$, respectivamente. Ambos casos presentan obsrvaciones atípicas, pero en los votos
negativos estos casos muestran valores más extremos, por lo que la media y el desvío se
ven influenciados y adoptan también valores más altos.

\begin{figure}[h!]
    \centering
    \includegraphics[scale=0.5]{../visualizations/speech_by_vote.png}
    \caption{Distribución de votos en los distintos partidos políticos.}
    \label{fig-distrib-speech}
\end{figure}

Por último, al hacer foco en la longitud de los discursos pronunciados, vemos que su
distribución varía dependiendo de si los medimos en \textit{tokens} totales o únicos.
Un \textit{token} es una secuencia de caracteres que queremos considerar como un
grupo\footnote{\citet*{bird2009natural}}. Aquí este grupo constituye lo que denominamos
`palabra'. En promedio, los discursos emitidos tienen 418 palabras, con un desvío
estándar de 714 palabra; la mediana es de 11 y la moda, de 7. Sin embargo, estas medidas
reflejan las palabras totales utilizadas en cada intervención, sin considerar si se repiten
o no: cada ocurrencia de una palabra cuenta, sin importar si ya fue pronunciada en el mismo
discurso. Es por eso que, a fines comparativos, se tomaron también las medidas de centralidad
de los \textit{tokens} únicos, las cuales mostraron una media de 161 palabras por discurso
con un desvío de 245, una mediana de 11 y una moda de 1. La figura \ref{fig-distrib-tokens}
refleja este contraste. Como es posible observar, si bien en ambos casos los discursos
muestran \textit{outliers} respecto de su longitud, cuando la misma se mide en palabras
totales, se presenta una mayor cantidad de casos atípicos con valores más extremos que en
el caso en el que se miden \textit{tokens} únicos. En este último, un
$20\percentsign$ de los registros son \textit{outliers}, de los cuales solo el
$17\percentsign$ constituyen casos extremos, mientras que, al considerar
el total de palabras, un $22\percentsign$ de los datos resultan atípicos y,
de ellos, el $32\percentsign$ son extremos\footnote{Para este análisis se utilizó el
test de Tukey, que toma el rango intercuartil o \textit{IQR} (\textit{Q3-Q1}, donde
\textit{Q1} refiere al primer cuartil y \textit{Q3}, al terceo) y considera
valores atípicos leves a aquellos que se encuentran entre
$Q1 - (1.5 * IQR) < x < Q3 - (1.5 * IQR)$ y como extremos a los que están entre
$Q1 - (3 * IQR) < x < Q3 - (3 * IQR)$.}.
\todo[color=green!40,size=\tiny]{La longitud está menos delimitada, pero la temática
es específica. Entonces podría ser, cuando son tokens únicos no haya tanta dispersión
porque el tópico es uno y las palabras para referirnos a él están un poco más delimitadas
que la cantidad de veces que se puede usar una de estas mismas palabras al en un discurso.}

\begin{figure}[h!]
    \centering
    \includegraphics[scale=0.5]{../visualizations/distrib_tokens.png}
    \caption{Distribución de \textit{tokens} totales y únicos en los discursos pronunciados. En ambos gráficos,
    la línea azul indica el límite a partir del cual una observación se considera atípica leve
    ($IQR*1.5$) y la naranja, el límite a partir del cual se la consifera atípica extrema ($IQR*3$).}
    \label{fig-distrib-tokens}
\end{figure}


%\begin{table}[ht]
%\centering
%\begin{tabular}{ |c|c|c|c|c| }
%    \hline
%    Voto & Media & Desvío & Mediana & Moda \\
%    \hline\hline
%    Abstención & 1.00 & 0.00 & 1.00 & 1 \\
%    \hline
%    Ausente & 0.50 & 0.71 & 0.50 & 0 \\
%    \hline
%    Negativo & 3.03 & 8.43 & 1.00 & 1 \\
%    \hline
%    Positivo & 2.92 & 4.26 & 2.00 & 1 \\
%    \hline
%\end{tabular}
%\caption{Medidas de centralidad de la cantidad de discursos emitidos}
%\label{table-tokens}
%\end{table}



\section{Metodolog\'ia}\label{section-methods}

\subsection{Anotaci\'on}\label{subsection-methods-annotation}
Luego del proceso de anotaci\'on y correcci\'on de lemas y etiquetas gramaticales,
se extrajeron algunas m\'etricas de resumen y se realizaron gr\'aficos con el fin de
comprender qu\'e distribuci\'on presentaban dichas etiquetas utilizadas y cu\'antos de los
datos etiquetdados de forma autom\'atica requirieron alguna correcci\'on, ya fuera en el
lema asignado, ya en la etiqueta gramatical predicha.
\par
En total, se revisaron 8324 anotaciones\footnote{Cada anotaci\'on est\'a dada por
la ocurrencia \textit{cruda} de la palabra, la etiqueta (o etitquetas, si el modelo
predijo m\'as de una posible) de esa palabra en los discursos en los que fue vista y el
lema (o lemas, si se predijo m\'as de uno) para esa palabra en tales contextos.}.
De ellas, se despreciaron 211 cadenas de caracteres ($1.86\percentsign$)
\footnote{Remitirse a \ref{appendix-annotation} para un detalle de las palabras
despreciadas.}. De las palabras que se persistieron para el an\'alisis,
el $20.4\percentsign$ requiri\'o una correcci\'on en el lema predicho.
De estas correcciones, el $27.8\percentsign$ de los casos corresponden a
palabras para las cuales el modelo realiz\'o diferentes predicciones en los distintos
contextos de aparici\'on y una de estas predicciones fue acertada. Estos casos se
consideraron un error porque, si bien el modelo predijo al menos una vez el lema correcto,
no lo hizo de forma consistente. El $72.2\percentsign$ de los errores
restantes fueron casos en los que el modelo no logr\'o predecir el lema correcto en
ninguna situaci\'on.
Respecto de las etiquetas \textit{POS} se procedi\'o con un an\'alisis similar.
El $12.35\percentsign$ de las palabras etiquetadas de manera autom\'atica requiri\'o
una correcci\'on. De estos casos de error, el $71\percentsign$ de las palabras
constituyen casos en los cuales el modelo predijo distintas etiquetas en los contextos
observados y una de ellas result\'o ser la adecuada.
En el resto de los casos ($29\percentsign$) el modelo no fue capaz de
identificar la etiqueta adecuada en ninguna de sus predicciones.
\par
Al final del procedimiento se obtuvo un conjunto de 4889 \textit{tokens} \'unicos,
donde la unicidad hace referencia al lema y a la etiqueta gramatical asignada.
En el an\'alisis exploratorio realizado para la descripci\'on de los datos, la unicidad
de los \textit{tokens} estaba dada por la igualdad en la secuencia de caracteres
en la ocurrencia efectiva de la palabra: all\'i `suma' se considera
una sola vez, independientemente de si, en sus distintas ocurrencias, refiere al
verbo (en la tercera persona singular del verbo `sumar') o al nombre
(\textit{``acci\'on y efecto de sumar''}\footnote{\Citeauthor{rae23diccionario}.}). Al
agrupar los \textit{tokens} por lema y etiqueta gramatical, estas dos
ocurrencias son consideradas por separado, dado que se toma en cuenta
`suma (\textit{NOUN})' y `sumar (\textit{VERB})'
\footnote{\textit{`Noun'} y \textit{`verb'} refieren a las clases `nombre'
(o `sustantivo') y `verbo', respectivamente. Otras clases frecuentes son \textit{`adj'},
por `adjetivo', y \textit{`adv'}, por `adverbio'. Para una lista completa
de las posibles etiquetas, visitar: \url{https://universaldependencies.org/u/pos/}
[\'ultimo acceso: 15--10--2024].}. La figura \ref{fig-distrib-unique-tokens} refleja el contraste
en la longitud de discursos al ser medidos con ambos enfoques.
All\'i se puede observar que las longitudes medidas considerando lemas y etiquedas
\textit{POS} reflejan valores menores\footnote{Es preciso recordar aqu\'i
que, si bien el uso de etiquetas \textit{POS} puede llevar a considerar
como dos o m\'as palabras lo que, en el an\'alisis previo se consideraba
como una palabra \'unica, el uso de los lemas (en contraste con
las palabras que reflejan acciedentes morfol\'ogicos) reduce en gran
medida el vocabulario de los documentos.}.

\begin{figure}[h!]
\centering
\includegraphics[scale=0.45]{../visualizations/distrib_tokes_vs_lemma_pos/distrib_tokens_vs_lemma_pos.png}
\caption{Distribuci\'on \textit{tokens} \'unicos aplicando distintos criterios de unicidad.
En la figura de la izquierda los \textit{tokens} se miden en palabras y, en la de la
derecha, a lemas con su correspondiente etiqueta \textit{POS}. El eje de ordenadas
(eje \textit{y}) indica la cantidad de \textit{tokens} en cada caso.}
\label{fig-distrib-unique-tokens}
\end{figure}

\subsection{Selecci\'on de rasgos}\label{subsection-methods-features}
Durante la validaci\'on cruzada realizada para evaluar el rendimiento de los
distintos vectorizadores se fueron registrando varias m\'etricas de evaluaci\'on
para cada iteraci\'on como as\'i tambi\'en el tiempo requerido para el entrenamiento
en cada caso. La figura \ref{fig-results-features-fit-time} muestra el promedio y el desv\'io
est\'andar de este \'ultimo en las cinco iteraciones realizadas. Como es posible
apreciar, el vectorizado de proporciones con remoci\'on de \textit{stopwords}
provenientes de \textit{NLTK} es el que presenta un tiempo de entrenamiento promedio
menor, y tambi\'en un menor desv\'io. El vectorizador de \textit{TF-IDF} con
frecuencia de documentos natural, por otro lado, es el que mayores valores
exhibe tanto en la media como en el desv\'io est\'andar. La tabla \ref{table-appendix-fit-time}
presente en la secci\'on \ref{appendix-table-vectorizers} del Anexo detalla una
a una las medidas de centralidad.

\begin{figure}[h!]
    \centering
    \includegraphics[scale=0.6]{../visualizations/features/fit_time.png}
    \caption{Media y desvi\'o est\'andar del tiempo requerido (medido en segundos)
    para el entrenamiento de una regresi\'on log\'istica \textit{baseline}
    al utilizar distintos vecotrizadores siguiendo una estrategia de
    validaci\'on cruzada con cinco iteraciones.}
    \label{fig-results-features-fit-time}
\end{figure}

En cuanto a las m\'etricas de rendimiento,
la tabla \ref{table-results-vectorizers-val} ofrece el promedio de los valores
obtenidos en la validaci\'on cruzada para cada vectorizador. All\'i se puede ver que,
en la mayor\'ia de los casos, fue el vectorizador de \textit{TF-IDF} con logaritmo
de la frecuencia de documentos el que present\'o mejores resultados. Si bien el
vectorizador de proporciones con remoci\'on de \textit{stopwords} obtenidas por el
m\'etodo de Zipf se muestra superior a este en t\'erminos de precisi\'on, solo
lo hace a expensas de perder cobertura, lo que se conoce como la compensaci\'on o el
\textit{trade-off} entre ambas m\'etricas. De hecho, esta vectorizaci\'on es la que
presenta la peor cobertura de todos los m\'etodos evaluados. Y algo semejante ocurre
con el vectorizador basado en el \textit{ratio} de los \textit{log-odds} con
suavizado, que
muestra una mejor cobertura que el resto de los vectorizadores, pero una peor
precisi\'on, \textit{accuracy} y \textit{f1-macro}.

\begin{table}[h!]
    \centering
    \begin{adjustbox}{max width=\textwidth}
    \begin{tabular}{ *{7}{|c}| }
        \hline
        Vectorizador & \textit{Accuracy} & Precisi\'on & Cobertura & \textit{F1} & \textit{F1-weighted} & \textit{F1-macro} \\
        \hline\hline
        \makecell{Frecuencias\\absolutas} & 0.610 & 0.663 & 0.618 & 0.635 & 0.608 & 0.605 \\
        \hline
        Proporciones & 0.623 & 0.663 & 0.653 & 0.654 & 0.622 & 0.617 \\
        \hline
        \makecell{Proporciones\\($NLTK$)} & 0.623 & 0.663 & 0.653 & 0.654 & 0.622 & 0.617 \\
        \hline
        \makecell{Proporciones\\($Zipf$)} & 0.636 & \cellcolor{highlight-blue!60}0.726 & \cellcolor{highlight-orange!60}0.576 & 0.619 & 0.624 & 0.625 \\
        \hline
        \makecell{\textit{Ratio} de\\\textit{odds}} & 0.560 & 0.587 & 0.698 & \cellcolor{highlight-orange!60}0.611 & 0.518 & 0.506 \\
        \hline
        \makecell{\textit{Ratio} de\\\textit{log-odds}} & 0.560 & 0.587 & 0.698 & \cellcolor{highlight-orange!60}0.611 & \cellcolor{highlight-orange!60}0.451 & 0.506 \\
        \hline
        \makecell{\textit{Ratio} de\\\textit{log-odds}\\(suavizado)} & \cellcolor{highlight-orange!60}0.541 & \cellcolor{highlight-orange!60}0.555 & \cellcolor{highlight-blue!60}0.887 & 0.682 & 0.518 & \cellcolor{highlight-orange!60}0.419 \\
        \hline
        \textit{TF-IDF} & 0.572 & 0.595 & 0.731 & 0.630 & 0.521 & 0.506 \\
        \hline
        \makecell{\textit{TF-IDF}\\($log IDF$)} & \cellcolor{highlight-blue!60}0.667 & 0.699 & 0.721 & \cellcolor{highlight-blue!60}0.702 & \cellcolor{highlight-blue!60}0.663 & \cellcolor{highlight-blue!60}0.657 \\
        \hline
        \textit{Word scores} & 0.623 & 0.655 & 0.697 & 0.671 & 0.618 & 0.611 \\
        \hline
    \end{tabular}
    \end{adjustbox}
    \caption{Resultados obtenidos tras evaluar un modelo de
    regresi\'on log\'istica base utilizando
    vectorizadores basados en las distintas t\'ecnicas estad\'isticas.
    Los valores reflejan el rendimiento promedio de las cinco iteraciones
    de la validaci\'on cruzada.
    Las celdas resaltadas en azul corresponden a la estategia de vectorizaci\'on
    que obtuvo un mejor rendimiento promedio en cada
    m\'etrica de evaluaci\'on y las resaltadas en naranja, a la
    que obtuvo el peor rendimiento.}
    \label{table-results-vectorizers-val}
\end{table}

El vectorizador basado en \textit{TF-IDF (log IDF)} no solo presenta un mayor
\textit{accuracy} que el resto de las t\'ecnicas de vectorizaci\'on, sino que
tambi\'en muestra un mejor \textit{$F_{\beta}$ score} (con $\beta=1$).
Esta m\'etrica ofrece una representaci\'on sim\'etrica de la precisi\'on y la cobertura,
lo que nos indica que, si bien dicho vectorizador no es el que mejores valores
exhibe en estas m\'etricas, s\'i es el que mejor maneja la compensaci\'on
entre ambas. En este trabajo se decidi\'o utilizar $\beta=1$ porque no se busca
privilegiar ninguna de las dos por sobre la otra.
\par
Adicionalmente, se reportan el \textit{F1-macro} y \textit{F1-weighted}.
El primero consiste en un promedio del \textit{F1} calculado para ambas clases
predichas, lo que da una idea de la compensaci\'on de la precisi\'on y la
cobertura tomado en cuenta la clase $1$ (discursos positivos), por un lado, y
la clase $0$ (discursos negativos), por otro. No obstante, dado que ambas clases no
est\'an balancedas\footnote{Como se mencion\'o en la secci\'on
\ref{subsection-data-description}, el conjunto de datos presenta un $56\percentsign$
de discursos a favor y un $44\percentsign$ de discursos en contra.}, podr\'ia ser que,
al calcular el promedio, el buen rendimiento en la predicci\'on de una de las clases
enmascare la imperfecta predicci\'on de la otra. Es por esto
que tambi\'en se recurre al \textit{F1-weighted}, que multiplica el \textit{F1}
de cada clase por la proporci\'on de casos que esta presenta en el conjunto de datos,
de modo que el valor resultante es una medida ``pesada'' en relaci\'on a la representaci\'on
de cada clase. En la secci\'on \ref{appendix-plots-vectorizers} del Anexo pueden
apreciarse los gr\'aficos de estas m\'etricas para cada iteraci\'on de la validaci\'on
cruzada.


\subsection{Entrenamiento y evaluaci\'on}\label{subsection-methods-models}
Para la evaluación y comparación de los modelos entrenados con los
distintos hiperparámetros se recurrió a las mismas métricas que las
referidas en la sección \ref{subsection-results-features}, al evaluar
los posibles vectorizadores.
\par
En este caso, se encontró que los hiperparámetros $C=0.1$ y $penalty=l2$
arrojaron los mejores valores promedios considerando las cinco iteraciones
de la validación cruzada. La única excepción a esto es el resultado en la
métrica de precisión, en el que el modelo entrenado con $C=2$ y $penalty=l2$
obtiene un mejor resultado. No obstante, este modelo muestra los valores
más bajos de cobertura y \textit{F1}. Es notable que, si bien tiene el mayor
valor de precisión observado, esto no llega a compensar su bajo rendimiento
en la cobertura y de ahí que también exhiba el valor de \textit{F1} más bajo.
Con valores más moderados, los demás modelos logran un mejor balance entre
\textit{precision} y \textit{recall}.

\begin{table}[h!]
    \centering
    \begin{adjustbox}{max width=\textwidth}
    \begin{tabular}{ *{7}{|c}| }
    \hline
    Parámetros & \textit{Accuracy} & Precisión & Cobertura & \textit{F1} & \textit{F1-macro} & \textit{F1-weighted} \\
    \hline\hline
    $C=0.1, penalty=l2$ & \cellcolor{highlight-blue!60}0.698 & 0.709 & \cellcolor{highlight-blue!60}0.786  & \cellcolor{highlight-blue!60}0.743 & \cellcolor{highlight-blue!60}0.686 & \cellcolor{highlight-blue!60}0.692 \\
    \hline
    $C=0.5, penalty=l2$ & 0.648 & 0.715 & 0.628  & 0.665 & 0.644 & 0.647 \\
    \hline
    $C=1, penalty=l2$ & \cellcolor{highlight-orange!60}0.622 & \cellcolor{highlight-orange!60}0.695 & 0.583 & 0.6310 & \cellcolor{highlight-orange!60}0.620 & \cellcolor{highlight-orange!60}0.622 \\
    \hline
    $C=2, penalty=l2$ & 0.629 & \cellcolor{highlight-blue!60}0.717 & \cellcolor{highlight-orange!60}0.560 & \cellcolor{highlight-orange!60}0.624 & 0.626 & 0.626 \\
    \hline
\end{tabular}
\end{adjustbox}
\caption{Resultados obtenidos tras evaluar un modelo de Regresión Logística con
distintos hiperparámetos. Los valores reflejan el rendimiento promedio de las
cinco iteraciones de la validación cruzada. Las celdas resaltadas en azul
corresponden a conjunto de hiperparámetros que obtuvo un mejor rendimiento
promedio en cada móetrica de evaluación y las resaltadas en naranja, al
que obtuvo el peor rendimiento.}
\end{table}

Adicionalmente, se graficó la media, el desvío estándar y el valor
obtenido en cada \textit{split} de la métrica \textit{F1}
para cada conjunto de hiperparámetros. Aquí podemos observar que el conjunto
de mejores hiperparámetros no solo presenta un desvío menor que los otros
conjuntos sino que, además, su rendimiento es mejor al resto en todas las
iteraciones.

\begin{figure}[h!]
    \centering
    \includegraphics[scale=0.5]{../visualizations/parameters_selection/f1_by_split.png}
    \caption{Contraste}
    \label{fig}
\end{figure}

\begin{table}[h!]
    \centering
    \begin{adjustbox}{max width=\textwidth}
    \begin{tabular}{ *{5}{|c}| }
    \hline
    Clase & Precisión & Cobertura & \textit{F1} & \textit{Soport} \\
    \hline\hline
    0 (en contra) & 0.80 & 0.67 & 0.73 & 18 \\
    \hline
    1 (a favor) & 0.76 & 0.86 & 0.81  & 22 \\
    \hline\hline
    \textit{Accuray} & & & 0.78 & 40 \\
    \hline
    \textit{Macro AVG} & 0.78 & 0.77 & 0.77 & 40 \\
    \hline
    \textit{Weighted AVG} & 0.78 & 0.78 & 0.77 & 40 \\
    \hline
\end{tabular}
\end{adjustbox}
\caption{Resultado.}
\end{table}


\begin{figure}[h!]
    \centering
    \includegraphics[scale=0.5]{../visualizations/models/confussion_matrix.png}
    \caption{Contraste}
    \label{fig}
\end{figure}

\begin{figure}[h!]
    \centering
    \includegraphics[scale=0.5]{../visualizations/models/lr_feature_importance_barplot_log_proba.png}
    \caption{Contraste}
    \label{fig}
\end{figure}

\section{Resultados}\label{section-results}

\subsection{Anotaci\'on}\label{subsec-results-annotation}
Luego del proceso de anotaci\'on y correcci\'on de lemas y etiquetas gramaticales,
se extrajeron algunas m\'etricas de resumen y se realizaron gr\'aficos con el fin de
comprender qu\'e distribuci\'on presentaban dichas etiquetas utilizadas y cu\'antos de los
datos etiquetdados de forma autom\'atica requirieron alguna correcci\'on, ya fuera en el
lema asignado, ya en la etiqueta gramatical predicha.
\par
En total, se revisaron 8324 anotaciones\footnote{Cada anotaci\'on est\'a dada por
la ocurrencia \textit{cruda} de la palabra, la etiqueta (o etitquetas, si el modelo
predijo m\'as de una posible) de esa palabra en los discursos en los que fue vista y el
lema (o lemas, si se predijo m\'as de uno) para esa palabra en tales contextos.}.
De ellas, se despreciaron 211 cadenas de caracteres ($1.86\percentsign$)
\footnote{Remitirse a \ref{appendix-annotation} para un detalle de las palabras
despreciadas.}. De las palabras que se persistieron para el an\'alisis,
el $20.4\percentsign$ requiri\'o una correcci\'on en el lema predicho.
De estas correcciones, el $27.8\percentsign$ de los casos corresponden a
palabras para las cuales el modelo realiz\'o diferentes predicciones en los distintos
contextos de aparici\'on y una de estas predicciones fue acertada. Estos casos se
consideraron un error porque, si bien el modelo predijo al menos una vez el lema correcto,
no lo hizo de forma consistente. El $72.2\percentsign$ de los errores
restantes fueron casos en los que el modelo no logr\'o predecir el lema correcto en
ninguna situaci\'on.
Respecto de las etiquetas \textit{POS} se procedi\'o con un an\'alisis similar.
El $12.35\percentsign$ de las palabras etiquetadas de manera autom\'atica requiri\'o
una correcci\'on. De estos casos de error, el $71\percentsign$ de las palabras
constituyen casos en los cuales el modelo predijo distintas etiquetas en los contextos
observados y una de ellas result\'o ser la adecuada.
En el resto de los casos ($29\percentsign$) el modelo no fue capaz de
identificar la etiqueta adecuada en ninguna de sus predicciones.
\par
Al final del procedimiento se obtuvo un conjunto de 4889 \textit{tokens} \'unicos,
donde la unicidad hace referencia al lema y a la etiqueta gramatical asignada.
En el an\'alisis exploratorio realizado para la descripci\'on de los datos, la unicidad
de los \textit{tokens} estaba dada por la igualdad en la secuencia de caracteres
en la ocurrencia efectiva de la palabra: all\'i `suma' se considera
una sola vez, independientemente de si, en sus distintas ocurrencias, refiere al
verbo (en la tercera persona singular del verbo `sumar') o al nombre
(\textit{``acci\'on y efecto de sumar''}\footnote{\Citeauthor{rae23diccionario}.}). Al
agrupar los \textit{tokens} por lema y etiqueta gramatical, estas dos
ocurrencias son consideradas por separado, dado que se toma en cuenta
`suma (\textit{NOUN})' y `sumar (\textit{VERB})'
\footnote{\textit{`Noun'} y \textit{`verb'} refieren a las clases `nombre'
(o `sustantivo') y `verbo', respectivamente. Otras clases frecuentes son \textit{`adj'},
por `adjetivo', y \textit{`adv'}, por `adverbio'. Para una lista completa
de las posibles etiquetas, visitar: \url{https://universaldependencies.org/u/pos/}
[\'ultimo acceso: 15--10--2024].}. La figura \ref{fig-distrib-unique-tokens} refleja el contraste
en la longitud de discursos al ser medidos con ambos enfoques.
All\'i se puede observar que las longitudes medidas considerando lemas y etiquedas
\textit{POS} reflejan valores menores\footnote{Es preciso recordar aqu\'i
que, si bien el uso de etiquetas \textit{POS} puede llevar a considerar
como dos o m\'as palabras lo que, en el an\'alisis previo se consideraba
como una palabra \'unica, el uso de los lemas (en contraste con
las palabras que reflejan acciedentes morfol\'ogicos) reduce en gran
medida el vocabulario de los documentos.}.

\begin{figure}[h!]
\centering
\includegraphics[scale=0.45]{../visualizations/distrib_tokes_vs_lemma_pos/distrib_tokens_vs_lemma_pos.png}
\caption{Distribuci\'on \textit{tokens} \'unicos aplicando distintos criterios de unicidad.
En la figura de la izquierda los \textit{tokens} se miden en palabras y, en la de la
derecha, a lemas con su correspondiente etiqueta \textit{POS}. El eje de ordenadas
(eje \textit{y}) indica la cantidad de \textit{tokens} en cada caso.}
\label{fig-distrib-unique-tokens}
\end{figure}

\subsection{Selecci\'on de rasgos}\label{subsection-results-features}
Durante la validaci\'on cruzada realizada para evaluar el rendimiento de los
distintos vectorizadores se fueron registrando varias m\'etricas de evaluaci\'on
para cada iteraci\'on como as\'i tambi\'en el tiempo requerido para el entrenamiento
en cada caso. La figura \ref{fig-results-features-fit-time} muestra el promedio y el desv\'io
est\'andar de este \'ultimo en las cinco iteraciones realizadas. Como es posible
apreciar, el vectorizado de proporciones con remoci\'on de \textit{stopwords}
provenientes de \textit{NLTK} es el que presenta un tiempo de entrenamiento promedio
menor, y tambi\'en un menor desv\'io. El vectorizador de \textit{TF-IDF} con
frecuencia de documentos natural, por otro lado, es el que mayores valores
exhibe tanto en la media como en el desv\'io est\'andar. La tabla \ref{table-appendix-fit-time}
presente en la secci\'on \ref{appendix-table-vectorizers} del Anexo detalla una
a una las medidas de centralidad.

\begin{figure}[h!]
    \centering
    \includegraphics[scale=0.6]{../visualizations/features/fit_time.png}
    \caption{Media y desvi\'o est\'andar del tiempo requerido (medido en segundos)
    para el entrenamiento de una regresi\'on log\'istica \textit{baseline}
    al utilizar distintos vecotrizadores siguiendo una estrategia de
    validaci\'on cruzada con cinco iteraciones.}
    \label{fig-results-features-fit-time}
\end{figure}

En cuanto a las m\'etricas de rendimiento,
la tabla \ref{table-results-vectorizers-val} ofrece el promedio de los valores
obtenidos en la validaci\'on cruzada para cada vectorizador. All\'i se puede ver que,
en la mayor\'ia de los casos, fue el vectorizador de \textit{TF-IDF} con logaritmo
de la frecuencia de documentos el que present\'o mejores resultados. Si bien el
vectorizador de proporciones con remoci\'on de \textit{stopwords} obtenidas por el
m\'etodo de Zipf se muestra superior a este en t\'erminos de precisi\'on, solo
lo hace a expensas de perder cobertura, lo que se conoce como la compensaci\'on o el
\textit{trade-off} entre ambas m\'etricas. De hecho, esta vectorizaci\'on es la que
presenta la peor cobertura de todos los m\'etodos evaluados. Y algo semejante ocurre
con el vectorizador basado en el \textit{ratio} de los \textit{log-odds} con
suavizado, que
muestra una mejor cobertura que el resto de los vectorizadores, pero una peor
precisi\'on, \textit{accuracy} y \textit{f1-macro}.

\begin{table}[h!]
    \centering
    \begin{adjustbox}{max width=\textwidth}
    \begin{tabular}{ *{7}{|c}| }
        \hline
        Vectorizador & \textit{Accuracy} & Precisi\'on & Cobertura & \textit{F1} & \textit{F1-weighted} & \textit{F1-macro} \\
        \hline\hline
        \makecell{Frecuencias\\absolutas} & 0.610 & 0.663 & 0.618 & 0.635 & 0.608 & 0.605 \\
        \hline
        Proporciones & 0.623 & 0.663 & 0.653 & 0.654 & 0.622 & 0.617 \\
        \hline
        \makecell{Proporciones\\($NLTK$)} & 0.623 & 0.663 & 0.653 & 0.654 & 0.622 & 0.617 \\
        \hline
        \makecell{Proporciones\\($Zipf$)} & 0.636 & \cellcolor{highlight-blue!60}0.726 & \cellcolor{highlight-orange!60}0.576 & 0.619 & 0.624 & 0.625 \\
        \hline
        \makecell{\textit{Ratio} de\\\textit{odds}} & 0.560 & 0.587 & 0.698 & \cellcolor{highlight-orange!60}0.611 & 0.518 & 0.506 \\
        \hline
        \makecell{\textit{Ratio} de\\\textit{log-odds}} & 0.560 & 0.587 & 0.698 & \cellcolor{highlight-orange!60}0.611 & \cellcolor{highlight-orange!60}0.451 & 0.506 \\
        \hline
        \makecell{\textit{Ratio} de\\\textit{log-odds}\\(suavizado)} & \cellcolor{highlight-orange!60}0.541 & \cellcolor{highlight-orange!60}0.555 & \cellcolor{highlight-blue!60}0.887 & 0.682 & 0.518 & \cellcolor{highlight-orange!60}0.419 \\
        \hline
        \textit{TF-IDF} & 0.572 & 0.595 & 0.731 & 0.630 & 0.521 & 0.506 \\
        \hline
        \makecell{\textit{TF-IDF}\\($log IDF$)} & \cellcolor{highlight-blue!60}0.667 & 0.699 & 0.721 & \cellcolor{highlight-blue!60}0.702 & \cellcolor{highlight-blue!60}0.663 & \cellcolor{highlight-blue!60}0.657 \\
        \hline
        \textit{Word scores} & 0.623 & 0.655 & 0.697 & 0.671 & 0.618 & 0.611 \\
        \hline
    \end{tabular}
    \end{adjustbox}
    \caption{Resultados obtenidos tras evaluar un modelo de
    regresi\'on log\'istica base utilizando
    vectorizadores basados en las distintas t\'ecnicas estad\'isticas.
    Los valores reflejan el rendimiento promedio de las cinco iteraciones
    de la validaci\'on cruzada.
    Las celdas resaltadas en azul corresponden a la estategia de vectorizaci\'on
    que obtuvo un mejor rendimiento promedio en cada
    m\'etrica de evaluaci\'on y las resaltadas en naranja, a la
    que obtuvo el peor rendimiento.}
    \label{table-results-vectorizers-val}
\end{table}

El vectorizador basado en \textit{TF-IDF (log IDF)} no solo presenta un mayor
\textit{accuracy} que el resto de las t\'ecnicas de vectorizaci\'on, sino que
tambi\'en muestra un mejor \textit{$F_{\beta}$ score} (con $\beta=1$).
Esta m\'etrica ofrece una representaci\'on sim\'etrica de la precisi\'on y la cobertura,
lo que nos indica que, si bien dicho vectorizador no es el que mejores valores
exhibe en estas m\'etricas, s\'i es el que mejor maneja la compensaci\'on
entre ambas. En este trabajo se decidi\'o utilizar $\beta=1$ porque no se busca
privilegiar ninguna de las dos por sobre la otra.
\par
Adicionalmente, se reportan el \textit{F1-macro} y \textit{F1-weighted}.
El primero consiste en un promedio del \textit{F1} calculado para ambas clases
predichas, lo que da una idea de la compensaci\'on de la precisi\'on y la
cobertura tomado en cuenta la clase $1$ (discursos positivos), por un lado, y
la clase $0$ (discursos negativos), por otro. No obstante, dado que ambas clases no
est\'an balancedas\footnote{Como se mencion\'o en la secci\'on
\ref{subsection-data-description}, el conjunto de datos presenta un $56\percentsign$
de discursos a favor y un $44\percentsign$ de discursos en contra.}, podr\'ia ser que,
al calcular el promedio, el buen rendimiento en la predicci\'on de una de las clases
enmascare la imperfecta predicci\'on de la otra. Es por esto
que tambi\'en se recurre al \textit{F1-weighted}, que multiplica el \textit{F1}
de cada clase por la proporci\'on de casos que esta presenta en el conjunto de datos,
de modo que el valor resultante es una medida ``pesada'' en relaci\'on a la representaci\'on
de cada clase. En la secci\'on \ref{appendix-plots-vectorizers} del Anexo pueden
apreciarse los gr\'aficos de estas m\'etricas para cada iteraci\'on de la validaci\'on
cruzada.


\subsection{Entrenamiento y evaluaci\'on}\label{subsection-results-models}
Para la evaluación y comparación de los modelos entrenados con los
distintos hiperparámetros se recurrió a las mismas métricas que las
referidas en la sección \ref{subsection-results-features}, al evaluar
los posibles vectorizadores.
\par
En este caso, se encontró que los hiperparámetros $C=0.1$ y $penalty=l2$
arrojaron los mejores valores promedios considerando las cinco iteraciones
de la validación cruzada. La única excepción a esto es el resultado en la
métrica de precisión, en el que el modelo entrenado con $C=2$ y $penalty=l2$
obtiene un mejor resultado. No obstante, este modelo muestra los valores
más bajos de cobertura y \textit{F1}. Es notable que, si bien tiene el mayor
valor de precisión observado, esto no llega a compensar su bajo rendimiento
en la cobertura y de ahí que también exhiba el valor de \textit{F1} más bajo.
Con valores más moderados, los demás modelos logran un mejor balance entre
\textit{precision} y \textit{recall}.

\begin{table}[h!]
    \centering
    \begin{adjustbox}{max width=\textwidth}
    \begin{tabular}{ *{7}{|c}| }
    \hline
    Parámetros & \textit{Accuracy} & Precisión & Cobertura & \textit{F1} & \textit{F1-macro} & \textit{F1-weighted} \\
    \hline\hline
    $C=0.1, penalty=l2$ & \cellcolor{highlight-blue!60}0.698 & 0.709 & \cellcolor{highlight-blue!60}0.786  & \cellcolor{highlight-blue!60}0.743 & \cellcolor{highlight-blue!60}0.686 & \cellcolor{highlight-blue!60}0.692 \\
    \hline
    $C=0.5, penalty=l2$ & 0.648 & 0.715 & 0.628  & 0.665 & 0.644 & 0.647 \\
    \hline
    $C=1, penalty=l2$ & \cellcolor{highlight-orange!60}0.622 & \cellcolor{highlight-orange!60}0.695 & 0.583 & 0.6310 & \cellcolor{highlight-orange!60}0.620 & \cellcolor{highlight-orange!60}0.622 \\
    \hline
    $C=2, penalty=l2$ & 0.629 & \cellcolor{highlight-blue!60}0.717 & \cellcolor{highlight-orange!60}0.560 & \cellcolor{highlight-orange!60}0.624 & 0.626 & 0.626 \\
    \hline
\end{tabular}
\end{adjustbox}
\caption{Resultados obtenidos tras evaluar un modelo de Regresión Logística con
distintos hiperparámetos. Los valores reflejan el rendimiento promedio de las
cinco iteraciones de la validación cruzada. Las celdas resaltadas en azul
corresponden a conjunto de hiperparámetros que obtuvo un mejor rendimiento
promedio en cada móetrica de evaluación y las resaltadas en naranja, al
que obtuvo el peor rendimiento.}
\end{table}

Adicionalmente, se graficó la media, el desvío estándar y el valor
obtenido en cada \textit{split} de la métrica \textit{F1}
para cada conjunto de hiperparámetros. Aquí podemos observar que el conjunto
de mejores hiperparámetros no solo presenta un desvío menor que los otros
conjuntos sino que, además, su rendimiento es mejor al resto en todas las
iteraciones.

\begin{figure}[h!]
    \centering
    \includegraphics[scale=0.5]{../visualizations/parameters_selection/f1_by_split.png}
    \caption{Contraste}
    \label{fig}
\end{figure}

\begin{table}[h!]
    \centering
    \begin{adjustbox}{max width=\textwidth}
    \begin{tabular}{ *{5}{|c}| }
    \hline
    Clase & Precisión & Cobertura & \textit{F1} & \textit{Soport} \\
    \hline\hline
    0 (en contra) & 0.80 & 0.67 & 0.73 & 18 \\
    \hline
    1 (a favor) & 0.76 & 0.86 & 0.81  & 22 \\
    \hline\hline
    \textit{Accuray} & & & 0.78 & 40 \\
    \hline
    \textit{Macro AVG} & 0.78 & 0.77 & 0.77 & 40 \\
    \hline
    \textit{Weighted AVG} & 0.78 & 0.78 & 0.77 & 40 \\
    \hline
\end{tabular}
\end{adjustbox}
\caption{Resultado.}
\end{table}


\begin{figure}[h!]
    \centering
    \includegraphics[scale=0.5]{../visualizations/models/confussion_matrix.png}
    \caption{Contraste}
    \label{fig}
\end{figure}

\begin{figure}[h!]
    \centering
    \includegraphics[scale=0.5]{../visualizations/models/lr_feature_importance_barplot_log_proba.png}
    \caption{Contraste}
    \label{fig}
\end{figure}

\section{Discusi\'on y conclusiones}\label{section-discussion}
En el trabajo llevado a cabo, se ha accedido de manera automatizada
a una fuente de datos p\'ublicos del Estado argentino como es
la base de sesiones taquigr\'aficas del Senado de la Naci\'on. Por medio
del uso de librer\'ias disponibles en el lenguaje de programaci\'on
\textit{Python 3} se ha transformado tales datos, se los ha etiquetado
recurriendo a modelos pre{--}entrenados y se los ha vectorizado
para entrenar un modelo de clasificación capaz de predecir si un
documento constituye un discurso a favor o en contra de la legalizaci\'on
del aborto, t\'opico tratado en la sesi\'on del Senado descargada.
\par
Durante el proceso de etiquetado, se halló que el modelo para
predecir lemas y etiquetas \textit{POS} mostr\'o un mejor rendimiento
en la predicción de estas últimas que en las primeras. No solo
porque fue necesario corregir manualmente una mayor cantidad de lemas
que de etiquetas \textit{POS}, sino porque, adem\'as, el porcentaje
de palabras en las cuales el modelo no acertó a predecir el lema
correcto en ninguna ocurrencia fue mayor al caso an\'alogo de etiquetas
\textit{POS}. 
\par
Por limitaciones de tiempo y recursos disponibles para el etiquetado,
se decidió simplificar casos como verbos reflexivos y cl\'iticos. En
estas situaciones los verbos fueron transcriptos en infnitivos y las
marcas reflexivas y de cliticidad fueron omitidas.
Una posible mejora a este procedimiento consiste en repensar si esas
decisiones fueron las mejores y qu\'e impacto tienen sobre los datos
(cu\'antos verbos se ven afectados y en qu\'e medida). Del mismo modo,
podr\'ia ser de inter\'es contar con m\'as de una anotaci\'on que
permitiera validar de forma m\'as robusta la calidad de las etiquetas
obtenidas. Para esto, lo \'optimo ser\'ia contar con anotaciones
humanas que siguieran un protocolo estandarizado\footnote{Las gu\'ias
de anotaci\'on elaboradas en este trabajo persiguen un prop\'osito
semejante. Si bien aqu\'i las anotaciones fueron realizadas por una sola
persona, estas gu\'ias proporcionaron uns forma de fijar reglas que
pudieran ser recordadas en las distintas sesiones de anotaci\'on.}, dado
que esto permite un mejor control de los criterios de etiquetas, las 
etiquetas a utilizar y la fiabilidad del procedimiento. Sin embargo,
contar con anotaciones provenientes de distintos algoritmos de etiquetado
también puede ser provechoso, y los resultados podr\'ian ser usados
para comparar estos algoritmos.
\par
Para la vectorizaci\'on, se evaluaron distintas t\'ecnicas posibles,
tomadas de \cite{monroe2008fightin}, con las cuales se procedi\'o
a seleccionar un conjunto de \textit{tokens} o palabras que luego
se usaron como rasgos para caracterizar los documentos a clasificar.
Si bien, en principio, se busc\'o hallar alternativas al usual
\textit{TF-IDF}, finalmente fue este
método el que arroj\'o mejores resultados, específicamente con la
versi\'on que emplea logaritmo sobre la frecuencia inversa en los
documentos.
\par
Durante la evaluaci\'on de los vectorizadores, se fij\'o en $300$
el n\'umero de dimensiones a generar, que no es otra cosa que la
cantidad de palabras consideradas como relevantes para caracterizar
ambos grupos de discursos a predecir. Podr\'ia ser se inter\'es
contrastar si se observan los mismos resultados al disminuir o
incrementar el n\'umero de dimensiones. Asimismo, en la implementación
desarrollada, las t\'ecnicas estad\'isticas se emplean solamente
para la selecci\'on de palabras pero, una vez seleccionadas, todos
los vectorizadores siguen el mismo procedimiento: cuentan la frecuencia
absoluta de tales palabras en los documentos. En una futura iteraci\'on
sobre este trabajo, se podr\'ian implementar vectorizadores que
no solo utilizasen tales t\'ecnicas estad\'isticas en la selecci\'on
de palabras sino tambi\'en en la asignación de pesos asignados
en los vectores.
\par
Respecto del modelo de clasificaci\'on entrenado, se opt\'o
por una Regresi\'on Log\'istica, la cual mostr\'o un buen
rendimiento y un ajuste a los datos de entrenamiento lo suficientemente
flexible como para no quedar muy circunscripta a ellos y poder
generalizar ante casos no vistos previamente. Los modelos de
Regresi\'on Log\'istica constituyen modelos discriminativos, es decir
que aprenden rasgos que les permiten diferenciar entre las clases
a predecir, aunque estos rasgosno sean necesariamente intrínsecos
a las clases mismas. De ser posible, sería de interés probar
un entrenamiento similar en algún modelo generativo como, por ejemplo
Naïve Bayes, y contrastar qu\'e rasgos resultan seleccionados en ese caso.
\par
Por \'ultimo, cabe notar que los datos utilizados en este trabajo
no han sido voluminosos. Esto se debe a que, tras la obtenci\'on
de la versi\'on taquigr\'afica de la sesi\'on, fue necesario un
meticuloso procedimiento de limpieza y preprocesamiento que permitiese
hacer uso de esos datos. Incluir otras sesiones hubiese implicado
una labor mucho mayor y las limitaciones de tiempo y recursos no
lo permitieron. Sin embargo, ser\'ia deseable aumentar, en trabajos
futuros, los datos sobre los cuales se han desarrollados los procedimientos
hasta aqu\'i detallados, de modo que los resultados puedan verificarse en
\textit{corpora} que no solo sean mayores en tamaño sino que, adem\'as,
aborden otras tem\'aticas y, en ese sentido, presenten una mayor
diversidad.

\newpage
\bibliographystyle{linquiry3}
\bibliography{bibliography}

\clearpage
\appendix
\section{Anexo}\label{appendix}

\subsection{Gu\'ia de anotaci\'on}\label{appendix-annotation}
Luego del proceso de anotaci\'on y correcci\'on de lemas y etiquetas gramaticales,
se extrajeron algunas m\'etricas de resumen y se realizaron gr\'aficos con el fin de
comprender qu\'e distribuci\'on presentaban dichas etiquetas utilizadas y cu\'antos de los
datos etiquetdados de forma autom\'atica requirieron alguna correcci\'on, ya fuera en el
lema asignado, ya en la etiqueta gramatical predicha.
\par
En total, se revisaron 8324 anotaciones\footnote{Cada anotaci\'on est\'a dada por
la ocurrencia \textit{cruda} de la palabra, la etiqueta (o etitquetas, si el modelo
predijo m\'as de una posible) de esa palabra en los discursos en los que fue vista y el
lema (o lemas, si se predijo m\'as de uno) para esa palabra en tales contextos.}.
De ellas, se despreciaron 211 cadenas de caracteres ($1.86\percentsign$)
\footnote{Remitirse a \ref{appendix-annotation} para un detalle de las palabras
despreciadas.}. De las palabras que se persistieron para el an\'alisis,
el $20.4\percentsign$ requiri\'o una correcci\'on en el lema predicho.
De estas correcciones, el $27.8\percentsign$ de los casos corresponden a
palabras para las cuales el modelo realiz\'o diferentes predicciones en los distintos
contextos de aparici\'on y una de estas predicciones fue acertada. Estos casos se
consideraron un error porque, si bien el modelo predijo al menos una vez el lema correcto,
no lo hizo de forma consistente. El $72.2\percentsign$ de los errores
restantes fueron casos en los que el modelo no logr\'o predecir el lema correcto en
ninguna situaci\'on.
Respecto de las etiquetas \textit{POS} se procedi\'o con un an\'alisis similar.
El $12.35\percentsign$ de las palabras etiquetadas de manera autom\'atica requiri\'o
una correcci\'on. De estos casos de error, el $71\percentsign$ de las palabras
constituyen casos en los cuales el modelo predijo distintas etiquetas en los contextos
observados y una de ellas result\'o ser la adecuada.
En el resto de los casos ($29\percentsign$) el modelo no fue capaz de
identificar la etiqueta adecuada en ninguna de sus predicciones.
\par
Al final del procedimiento se obtuvo un conjunto de 4889 \textit{tokens} \'unicos,
donde la unicidad hace referencia al lema y a la etiqueta gramatical asignada.
En el an\'alisis exploratorio realizado para la descripci\'on de los datos, la unicidad
de los \textit{tokens} estaba dada por la igualdad en la secuencia de caracteres
en la ocurrencia efectiva de la palabra: all\'i `suma' se considera
una sola vez, independientemente de si, en sus distintas ocurrencias, refiere al
verbo (en la tercera persona singular del verbo `sumar') o al nombre
(\textit{``acci\'on y efecto de sumar''}\footnote{\Citeauthor{rae23diccionario}.}). Al
agrupar los \textit{tokens} por lema y etiqueta gramatical, estas dos
ocurrencias son consideradas por separado, dado que se toma en cuenta
`suma (\textit{NOUN})' y `sumar (\textit{VERB})'
\footnote{\textit{`Noun'} y \textit{`verb'} refieren a las clases `nombre'
(o `sustantivo') y `verbo', respectivamente. Otras clases frecuentes son \textit{`adj'},
por `adjetivo', y \textit{`adv'}, por `adverbio'. Para una lista completa
de las posibles etiquetas, visitar: \url{https://universaldependencies.org/u/pos/}
[\'ultimo acceso: 15--10--2024].}. La figura \ref{fig-distrib-unique-tokens} refleja el contraste
en la longitud de discursos al ser medidos con ambos enfoques.
All\'i se puede observar que las longitudes medidas considerando lemas y etiquedas
\textit{POS} reflejan valores menores\footnote{Es preciso recordar aqu\'i
que, si bien el uso de etiquetas \textit{POS} puede llevar a considerar
como dos o m\'as palabras lo que, en el an\'alisis previo se consideraba
como una palabra \'unica, el uso de los lemas (en contraste con
las palabras que reflejan acciedentes morfol\'ogicos) reduce en gran
medida el vocabulario de los documentos.}.

\begin{figure}[h!]
\centering
\includegraphics[scale=0.45]{../visualizations/distrib_tokes_vs_lemma_pos/distrib_tokens_vs_lemma_pos.png}
\caption{Distribuci\'on \textit{tokens} \'unicos aplicando distintos criterios de unicidad.
En la figura de la izquierda los \textit{tokens} se miden en palabras y, en la de la
derecha, a lemas con su correspondiente etiqueta \textit{POS}. El eje de ordenadas
(eje \textit{y}) indica la cantidad de \textit{tokens} en cada caso.}
\label{fig-distrib-unique-tokens}
\end{figure}

\subsection{Tablas adicionales}\label{appendix-tables}
\subsubsection{Selecci\'on de vectorizador}
\label{appendix-table-vectorizers}

\begin{table}[!htb]
    \centering
    \scalebox{0.8}{
    \begin{tabular}{ |c|c|c|c|c|c|c| }
    \hline
    Vectorizador & Media & Mediana & Desv\'io & M\'aximo \\
    \hline\hline
    Frecuencias absolutas & 0.190 & 0.153 & 0.063 & 0.284 \\
    \hline
    Proporciones & 0.198 & 0.144 & 0.128 & 0.428 \\
    \hline
    Proporciones ($NLTK$) & \cellcolor{highlight-blue!60}0.142 & \cellcolor{highlight-blue!60}0.142 & \cellcolor{highlight-blue!60}0.009 & \cellcolor{highlight-blue!60}0.155 \\
    \hline
    Proporciones ($Zipf$) & 0.249 & 0.170 & 0.147 & 0.449 \\
    \hline
    \textit{Ratio} de \textit{log odds} & 0.184 & 0.148 & 0.088 & 0.341 \\
    \hline
    \textit{Ratio} de \textit{log odds} (suavizado) & 0.286 & 0.258 & 0.141 & 0.470 \\
    \hline
    \textit{Ratio} de \textit{odds} & 0.189 & 0.148 & 0.094 & 0.355 \\
    \hline
    \textit{TF-IDF} & \cellcolor{highlight-orange!60}0.494 & \cellcolor{highlight-orange!60}0.316 & \cellcolor{highlight-orange!60}0.285 & \cellcolor{highlight-orange!60}0.867 \\
    \hline
    \textit{TF-IDF} (\textit{log IDF}) & 0.422 & 0.289 & 0.199 & 0.644 \\
    \hline
    \textit{Word scores} & 0.304 & 0.228 & 0.143 & 0.484 \\
    \hline
\end{tabular}}
\caption{Resultados obtenidos tras vectorizar los discursos analizados
con las distintas t\'ecnicas estad\'isticas y luego entrenar un modelo de
Regresi\'on Log\'istica \textit{baseline}. Las celdas resaltadas en azul corresponden
a la estategia de vectorizaci\'on que obetuvo un mejor rendimiento en cada
m\'etrica de evaluaci\'on.}
\label{table-appendix-fit-time}
\end{table}
%\FloatBarrier



   

\subsection{Gr\'aficos adicionales}\label{appendix-plots}
Aqu\'i pueden encontrarse gr\'aficos adicionales sobre los m\'etodos empleados
en este trabajo.

\subsubsection{Ley de Zipf}
\label{appendix-plots-zipf-law}

\begin{figure}[!htb]
    \centering
    \includegraphics[scale=0.4]{../visualizations/ley_de_zipf.png}
    \caption{Contraste entre la frecuencia absoluta de cada palabra (en logaritmo)
    y el \textit{ranking} de esa palabra seg\'un su frecuencia. El gr\'afico de la
    izquierda exhibe el orden natural de cada palabra y el de la derecha, su logaritmo.
    En ambos casos, la l\'inea punteada naranja indica el umbral
    para considerar \textit{stopword} a una palabra.}
    \label{fig-zipf-law}
\end{figure}
\FloatBarrier

\subsubsection{Selecci\'on de vectorizador}
\label{appendix-plots-vectorizers}

\begin{figure}[!htb]
    \centering
    \includegraphics[scale=0.4]{../visualizations/features/f1_by_split.png}
    \caption{\textit{F1} por split de validaci\'on cruzada para el conjunto
    de entrenmiento y de testeo, para todos los vectorizadores en evaluaci\'on.}
    \label{fig-vectorizers-f1}
\end{figure}
\FloatBarrier

\begin{figure}[!htb]
    \centering
    \includegraphics[scale=0.4]{../visualizations/features/f1_macro_by_split.png}
    \caption{\textit{F1-macro} por split de validaci\'on cruzada para el conjunto
    de entrenmiento y de testeo, para todos los vectorizadores en evaluaci\'on.}
    \label{fig-vectorizers-f1-macro}
\end{figure}
\FloatBarrier

\begin{figure}[!htb]
    \centering
    \includegraphics[scale=0.4]{../visualizations/features/f1_weighted_by_split.png}
    \caption{\textit{F1-weighted} o \textit{F1-pesado} por split de validaci\'on
    cruzada para el conjunto de entrenmiento y de testeo, para todos los
    vectorizadores en evaluaci\'on.}
    \label{fig-vectorizers-f1-weighted}
\end{figure}
\FloatBarrier

% cite title
%\usebibentry{monroe2008fightin}{title}


\end{document}
