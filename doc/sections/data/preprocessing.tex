Para este trabajo se utilizó la versión {taquigr\'afica} de la sesión número
23 (reunión 28) celebrada por la {C\'amara} de Senadores, en la que se abordó la
regulación del acceso a la interrupción voluntaria del embarazo y a sus
posteriores cuidados. Los datos fueron obtenidos de la {p\'agina} del Senado de la Nación
Argentina\footnote{\url{https://www.senado.gob.ar/parlamentario/sesiones/29-12-2020/28/downloadTac}}
utilizando el método de \textit{scraping}.\par
En primer lugar, se descargó la transcripción en formato \textit{PDF} y, luego, por
medio de la librería \textit{pdfminer}\footnote{\url{https://pdfminersix.readthedocs.io/en/latest/}},
se lo convirtió a texto plano, de manera que fuese procesable. Dado que la librería
empleada en la conversión no arrojó una transcripción limpia y que, {adem\'as}, no todo el texto
resultaba de relevancia para el presente trabajo, se {realiz\'o} una tarea de
preprocesamiento, con el fin de limpiar y organizar los datos de forma útil a los
objetivos aquí perseguidos.\par
Como primer paso del preprocesamiento se quitaron los encabezados y pies de paǵina.
El archivo \textit{.pdf}, convertido a \textit{.txt}, no contaba con separación de
{p\'aginas}, por lo que los encabezados y pies se convertían en líneas de texto que
interrumpían los discursos de los senadores y participantes del debate.
Posteriormente, se extrajo la sección del texto pertinente para este trabajo:
la sección 6, dedicada a la ``Regulación  del  acceso  a  la  interrupción
voluntaria  del  embarazo  y  a  la atención postaborto''. Otras secciones consistían
en el izamineto de la bandera, la convocatoria a la sesión, la lectura de la ley
resultante, entre otras cuestiones que no hacían al discurso argumentativos de los
asistentes, sino {m\'as} bien a cuestiones protocolares, por lo que fueron desestimadas
para este {an\'alisis}.\par
Con el texto de interés delimitado, y mediante el uso de patrones regulares, se
identificó a los distintos oradores y sus respectivos discursos, como así también
a las secciones que no pertenecían a fragmentos discursivos emitidos durante la
discusión, sino a comentarios agregados por el taquígrafo\footnote{``Luego de unos
instantes'' y ``Contenido no inteligible'' constituyen ejemplos de estos
fragmentos.}. Siguiendo la metodología descripta por \cite{monroe2008fightin},
los datos fueron guardados en un archivo en formato \textit{.xml}, en el que
cada fragmento fue etiquetado con las siguiente información:
\begin{itemize}
    \item \texttt{speech}: etiqueta que indica si el fragmento corresponde a un
    discurso efectivamente emitido durante la discusión, en cuyo caso toma el
    valor \texttt{true}, o si consiste en un comentario del taquígrafo, ante lo cual
    toma el valor \texttt{false}.
    \item \texttt{speaker}: en el caso de que el contenido del texto refiera a
    un fragmento discursivo, esta etiqueta indica el nombre del orador que lo
    pronunció. Si el fragmento es un comentario del traquígrafo, la etiqueta
    recibe el valor \texttt{none}.
\end{itemize}
\par
Adicionalmente, como resultado del preprocesamiento, se extrajo la lista de
asistentes a la sesión en formato tabular, con el objetivo de poder
enriquecerla adicionando especificaciónes como la filiación partidaria y la
decisión de voto. En el caso de la filiación partidaria, se utilizó la
librería \textit{selenium}\footnote{\url{https://selenium-python.readthedocs.io/index.html}}
para buscar, en la {p\'agina} del Senado destinada a tal
fin\footnote{\url{https://www.senado.gob.ar/senadores/Historico/Fecha}},
los senadores en ejercicio en la fecha en la que tuvo lugar la sesión.
Esta librería permite un interacción dinámica con la página \textit{web} consultada,
por lo que posibilitó el acceso al motor de búsqueda de la del Senado, la selección
del período deseado de forma automática y el procesamiento de los datos de interés,
los cuales se encontraban en \textit{.html} en la página pero que, luego de su
extracci\'on fueron guardados en formato \textit{.csv}.
\par
La decisón de voto, en cambio, debió ser transcripta de forma manual. Si bien
se encontraba detallada en el documento de la sesión descarcargado, no fue posible
convertir la tabla del documento \textit{.pdf} a texto plano sin pérdida de información.
Por lo cual, tomando los nombres de los senadores previamente extra\'idos, se gener\'o
un \textit{csv} en el cual se le agregó a cada uno el voto emitido.
\par
Una vez obtenidos los distintos conjuntos de datos para el estudio en cuestión,
fue necesario un paso ulterior de procesamiento a fin de que la información
recopilada de diversas fuentes pudiera ser vinculada de manera
satisfactoria. El desaf\'io aqu\'i residi\'o en la discrepancia entre
dichas fuentes para nombrar a los senadores. Al incio de la versión taquigráfica
es posible encontrar una lista de todos los asistentes a la sesión del Senado,
incluyendo no solo a los senadores con poder de voto sino también a quienes presidieron
y oficiaron en la secretaría. Esta lista, utilizada para la confección
de la tabla que vincula a cada senador con el voto emitido, enuncia a cada funcionario
recurriendo primero al nombre y luego, al apellido (en los caoss en los que
la persona tiene más de un nombre y/o apellido, se hizo uso de todos ellos). Esta
enunciación, sin embargo, es modificada a lo largo de la transcripción, donde la
primera vez que cada interlocutor hace uso de la palabra, se lo refiere con apellido
y nombre (en ese orden) y luego, solo con el apellido (uno solo en los casos que no
se prestan a confusión, o más de uno en caso contrario). Así también, utilizando primero
el apellido y luego, el nombre es como se hace referencia a los senadores en la página
destinada a proporcionar su lugar de origen y filiación partidaria. Para lidiar con esta
diversidad en la enunciación, se realizó un proceso de estandarización híbrido en el
cual, tomando como base los nombres y apellidos en un \textit{dataset}, se exploró
exhaustivamente con cuáles podían corresponderse en otro \textit{dataset}. Para esto, se
consideró cada cadena de nombre/s y apellido/s como un conjunto y se evaluó si existía
coincidencia entre ambos o si alguno era subconjunto de otro. En la mayoría de los casos,
este mecanismo permitió la desambiguación. Los pocos casos que no pudieron ser estandarizados
de esta manera por darse relación entre más de dos conjuntos fueron desambiguados manualmente.