Para este trabajo se utiliz\'o la versi\'on {taquigr\'afica} de la sesi\'on n\'umero
23 (reuni\'on 28) celebrada por la {C\'amara} de Senadores, que tuvo lugar entre los d\'ias
29 y 30 de diciembre de 2020 y en la que se abord\'o la
regulaci\'on del acceso a la interrupci\'on voluntaria del embarazo y a sus
posteriores cuidados. Los datos fueron obtenidos de la {p\'agina} del Senado de la Naci\'on
Argentina\footnote{\url{https://www.senado.gob.ar/parlamentario/sesiones/29-12-2020/28/downloadTac}
[\'ultimo acceso: 15--10--2024].}
utilizando el m\'etodo de \textit{scraping}.\par
Todo el c\'odigo de programaci\'on implementado fue desarrollado en el lenguaje
\textit{Python 3}. En primer lugar, se descarg\'o la transcripci\'on en formato \textit{PDF} y luego, por
medio de la librer\'ia \textit{pdfminer}\footnote{\url{https://pdfminersix.readthedocs.io/en/latest/}
[\'ultimo acceso: 15--10--2024].},
se la convirti\'o a texto plano de manera que fuese procesable. Dado que la librer\'ia
empleada en la conversi\'on no arroj\'o una transcripci\'on limpia y que, {adem\'as}, no todo el texto
result\'o de relevancia para el presente trabajo, se {realiz\'o} una tarea de
preprocesamiento con el fin de limpiar y organizar los datos de forma \'util a los
objetivos aqu\'i perseguidos.\par
Una vez convertida a \textit{.txt}, la transcripci\'on ya no present\'o separaci\'on de
{p\'aginas}, por lo que los encabezados y pies se convirtieron en l\'ineas de texto que
interrump\'ian los discursos de los participantes del debate y debieron
ser removidos.
Posteriormente, se extrajo la secci\'on del texto pertinente para este trabajo:
la secci\'on 6, dedicada a la ``Regulaci\'on  del  acceso  a  la  interrupci\'on
voluntaria  del  embarazo  y  a  la atenci\'on postaborto''. Otras secciones consist\'ian
en el izamineto de la bandera, la convocatoria a la sesi\'on, la lectura de la ley
resultante, entre otras cuestiones que no hac\'ian al discurso argumentativos de los
asistentes, sino {m\'as} bien al protocolo, por lo que fueron desestimadas
para este {an\'alisis}.\par
Con el texto de inter\'es delimitado y mediante el uso de patrones regulares, se
identific\'o a los distintos oradores y sus respectivos discursos, como as\'i tambi\'en
a las secciones que no pertenec\'ian a fragmentos discursivos emitidos durante la
discusi\'on, sino a comentarios agregados por el taqu\'igrafo\footnote{``Luego de unos
instantes'' y ``Contenido no inteligible'' constituyen ejemplos de estos
fragmentos.}. Siguiendo la metodolog\'ia descripta por \cite{monroe2008fightin},
los datos fueron guardados en un archivo en formato \textit{.xml}, en el que
cada fragmento fue etiquetado con la siguiente informaci\'on:
\begin{itemize}
    \item \texttt{speech}: etiqueta que indica si el fragmento corresponde
    propiamente a un discurso emitido durante la discusi\'on, en cuyo caso toma el
    valor \texttt{true}, o si consiste en un comentario del taqu\'igrafo, ante lo cual
    toma el valor \texttt{false}.
    \item \texttt{speaker}: en el caso de que el contenido del texto refiera a
    un fragmento discursivo, esta etiqueta indica el nombre del orador que lo
    pronunci\'o. Si el fragmento es un comentario del traqu\'igrafo, la etiqueta
    recibe el valor \texttt{none}.
\end{itemize}
\par
Adicionalmente, como resultado del preprocesamiento, se extrajo la lista de
asistentes a la sesi\'on en formato tabular, con el objetivo de poder
enriquecerla con especificaciones como la filiaci\'on partidaria y la
decisi\'on de voto. En el caso de la filiaci\'on partidaria, se utiliz\'o la
librer\'ia \textit{selenium}\footnote{\url{https://selenium-python.readthedocs.io/index.html}
[\'ultimo acceso: 15--10--2024].}
para buscar, en la {p\'agina} del Senado destinada a tal
fin\footnote{\url{https://www.senado.gob.ar/senadores/Historico/Fecha}
[\'ultimo acceso: 15--10--2024].},
los senadores en ejercicio en la fecha en la que tuvo lugar la sesi\'on.
Esta librer\'ia permite una interacci\'on din\'amica con la p\'agina \textit{web} consultada,
por lo que posibilit\'o el acceso al motor de b\'usqueda del sitio del Senado, la selecci\'on
del per\'iodo deseado de forma autom\'atica y el procesamiento de los datos de inter\'es,
los cuales se encontraban en \textit{.html} en la p\'agina pero que, luego de su
extracci\'on, fueron guardados en formato \textit{.csv}.
\par
La decis\'on de voto, en cambio, debi\'o ser transcripta de forma manual. Si bien
se encontraba detallada en el documento de la sesi\'on descarcargado, no fue posible
convertir la tabla del documento \textit{.pdf} a texto plano sin p\'erdida de informaci\'on.
Por lo cual, tomando los nombres de los senadores previamente extra\'idos, se gener\'o
un archivo \textit{.csv} en el cual se le agreg\'o a cada uno el voto emitido.
\par
Una vez obtenidos los distintos conjuntos de datos para el estudio en cuesti\'on,
fue necesario un paso ulterior de procesamiento a fin de que la informaci\'on
recopilada de diversas fuentes pudiera ser vinculada de manera
satisfactoria. El desaf\'io aqu\'i residi\'o en la discrepancia entre
dichas fuentes para nombrar a los senadores. Al incio de la versi\'on taquigr\'afica
es posible encontrar una lista de todos los asistentes a la sesi\'on del Senado,
incluyendo no solo a los senadores con poder de voto sino tambi\'en a quienes presidieron
y oficiaron en la secretar\'ia. Esta lista, utilizada para la confecci\'on
de la tabla que vincula a cada senador con el voto emitido, enuncia a cada funcionario
recurriendo primero al nombre y luego, al apellido (en los casos en los que
la persona tiene m\'as de un nombre y/o apellido, se hizo uso de todos ellos). Esta
enunciaci\'on, sin embargo, es modificada a lo largo de la transcripci\'on, donde la
primera vez que cada interlocutor hace uso de la palabra se lo refiere con apellido
y nombre (en ese orden) y luego, solo con el apellido (un solo apellido en los casos que no
se prestan a confusi\'on, o m\'as de uno en caso contrario). As\'i tambi\'en, utilizando primero
el apellido y luego el nombre es como se hace referencia a los senadores en la p\'agina
destinada a proporcionar su lugar de origen y filiaci\'on partidaria. Para lidiar con esta
diversidad en la enunciaci\'on, se realiz\'o un proceso de estandarizaci\'on h\'ibrido en el
cual, tomando como base los nombres y apellidos en un conjunto de datos, se explor\'o
exhaustivamente su correspondencia con con aquellos presentes en otro.
Para esto, se consider\'o cada cadena de nombre/s y apellido/s como un conjunto
y se evalu\'o la existencia de coincidencia
o de relaci\'on de subconjunto entre las distintas cadenas posibles. En la mayor\'ia de los casos,
este mecanismo permiti\'o la desambiguaci\'on. Los pocos casos que no pudieron ser estandarizados
de esta manera por darse relaci\'on entre m\'as de dos conjuntos fueron desambiguados manualmente.