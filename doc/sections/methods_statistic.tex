Siguiendo a \cite{monroe2008fightin}, se aplicó una serie de técnicas
no basadas en modelos probabilísticos para la selección de rasgos.
Dos de los métodos aplicados (\textit{TF-IDF} y \textit{Word Scores}) consisten
en algoritmos para la asignación de puntuaciones o \textit{scores} a las
palabras dentro de un \textit{corpus}. El resto de las técnicas utilizan
métricas estadísticas sencillas.

\subsubsection{Diferencia de frecuencias absolutas}
En primer lugar, se contrastaron las palabras de los discursos de los
senadores con votos positivos y las de los de senadores con votos negativos
utilizando la diferencia de frecuencias absolutas de estas palabras:

\begin{equation*}
y_w = y_{w}^{(P)}-y_{w}^{(N)}
\end{equation*}

En los casos en los que $y_{w}$ sea mayor o igual a $0$, la palabra
\textit{w} será característica del discurso de los senadores que votaron a
favor de la legalización. En caso contrario, será característica de quienes
votaron en contra.

\subsubsection{Diferencia de proporciones}
El segundo contraste realizado utilizó, en lugar de las frecuencias absolutas,
la proporción de las palabras en cada conjunto de discursos.

\begin{equation*}
    f_w = f_{w}^{(P)}-f_{w}^{(N)}
\end{equation*}


donde $f_{w}^{(X)}$ se define como $y_{w}^{(X)} / n^{(X)}$, siendo $y_{w}^{(X)}$
la frecuencia absoluta de la palabra $w$ en el cojunto los discursos de $X$
y $n^{(X)}$, la cantidad de palabras totales en esos discursos.

El mismo abordaje se intentó removiendo previamente las palabras consideradas
\textit{stopwords}. Para esto se utilizó el conjunto de palabras predefinidas
como tales de la librería \textit{NLTK}.
\footnote{\url{https://raw.githubusercontent.com/nltk/nltk_data/gh-pages/packages/corpora/stopwords.zip}}

\subsubsection{\textit{Odds}}
Asimismo, se compararon los \textit{odds} de ambos conjuntos de discursos:

\begin{equation*}
    O_w = \frac{O_{w}^{(P)}}{O_{w}^{(N)}}
\end{equation*}

aquí, $O_{w}^{(X)}$ consiste en $f_{w}^{(X)}/(1-f_{w}^{(X)})$.

\subsubsection{Ratio \textit{Log-odds}}
Sobre los \textit{odds}, también se calculó el \textit{ratio log-odds} y
se realizó una comparación con ellos. Para este cálculo se utilizó el
logaritmo natural, $\ln{O_w}$

\subsubsection{\textit{TF-IDF}}
Los autores del artículo ``\usebibentry{monroe2008fightin}{title}'' indican
que utilizaron también, como estadística comparativa, el algoritmo de asignación
de peso \textit{TF-IDF}. Siguiendo su proceder, en este trabajo se intentó
hacer lo mismo. Para ello, fue necesario implementar el cálculo, dado que
la fórmula utilizada en el \textit{paper} no era la misma que la implementada
por \textit{scikit-learn}, la librería usual para realizar este tipo extraer
este tipo de \textit{scores}.\footnote{Para más información sobre las fórmulas
implementadas por \textit{scikit-learn}, visitar:
\url{https://scikit-learn.org/stable/modules/feature_extraction.html\#tfidf-term-weighting}}\par
La fórmula usada por los autores, según ellos mismo señalan, es:

\begin{equation*}
    tf.idf_{w}^{X} = f_{w}^{X} \times \ln\bigg({\frac{1}{df_{w}}}\bigg)
\end{equation*}

donde $f_{w}^{X}$ refiere a la proporción de la palabra $w$ en el discurso
perteneciente a $X$ (con $X \in \lbrace P,N \rbrace$) y $df_w$ refiere a la
cantidad de documentos totales en los que aparece la palabra $w$.\par
Lo que los autores no mencionan es qué tipo de comparación realizan entre las
palabras pertenecientes a los distintos discursos. Siguiendo los procedimientos
anteriores, en este trabajo se optó por usar la sustracción, por lo que:
\todo[color=pink]{en resultados, recordar hablar de lo polémico de este abordaje
para}

\begin{equation*}
    tf.idf_{w} = tf.idf_{w}^{P}-tf.idf_{w}^{N}
\end{equation*}

\subsubsection{\textit{Word Scores}}
Por último, se utilizó también el procedimiento \textit{WordScores} descripto por
\cite{laver2003extracting}, que le asigna un peso a las palabras a partir
del siguiente cálculo:\todo[color=pink]{checkear si es de ellos o de quién}

\begin{equation*}
    W_{w}^{^*(P-N)} = \frac{y_{w}^{(P)}/n^{(P)}-y_{w}^{(N)}/n^{(N)}}{y_{w}^{(P)}/n^{(P)}+y_{w}^{(N)}/n^{(N)}}n_{k}
\end{equation*}\todo[color=pink]{leer lowe 2008, understanding wordscores y demostar
this is nearly identical to the difference of proportions measure (correlate at over +0.998)}
