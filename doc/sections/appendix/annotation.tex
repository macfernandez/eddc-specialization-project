Para todas las categorías de palabras se prefiere la lematización. A continuación
se detallan las decisiones tomadas en casos particulares.\todo{Indicar el sistema
de POS tag utilizado}

\subsubsection{Verbos}
\begin{itemize}
    \item Se omiten los clíticos (`la', `lo', `le' y sus formas plurales). En caso de
    que el verbo presente una forma con clítico, se lo omite y se lo convierte a su
    forma infinitiva. Ejemplo: `consierar\textit{la}' $\rightarrow$ `consierar'.
    \item Se omiten las formas reflexivas (`se'). En caso de que el verbo presente
    una forma reflexiva, se lo omite y se lo convierte a su forma infinitiva.
    Ejemplo: `enfrentar\textit{se}' $\rightarrow$ `enfrentar'.
\end{itemize}

\subsubsection{Nombres}
Por cuestiones de relevancia para el estudio, se mantiene la marca de género (femenino
o masculino), pero se omite la marca de número (singular o plural). Ejemplo: `abuelas'
$\rightarrow$ `abuela'.

\subsubsection{Números}
Se omiten todos, aquellos que aparecen en dígitos (`1', `10', etc.) y los que aparecen
en caracters (`cuarenta').

\subsubsection{Conectores}
Clases de palabra como preposiciones (`según'), pronombres relativos (`que') o
conjunciones (`si') que suelen funcionar como conectores subordinantes son
etiquetados con la categoría \textit{SCONJ}.

\subsubsection{Verbos auxiliares}
\begin{itemize}
    \item La categoría \textit{AUX} es despreciada.
    \item Aquellos verbos catalogados con la etiqueta \textit{AUX} fueron
    recatalogados como verbos (\textit{VERB}) si pertenecen a ese tipo de
    palabra, y transformados a la forma en infinitivo (`seamos' $\rightarrow$ `ser')
\end{itemize}

%### Palabras que se omiten
%
%- se indican con `FALSE` en la columna `use`
%- se omiten aquellas palabras que:
%  - son dígitos (`NUM`)
%  - son signos de puntuación (`PUNCT`)
%  - son palabras más signo de puntuación (`SYM`)
%  - son u operan como conectores (`SCONJ`, `CCONJ`)
