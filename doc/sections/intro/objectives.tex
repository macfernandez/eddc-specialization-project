En este trabajo y a partir del acceso a los datos de la C\'amara de Senadores,
se retoman las t\'ecnicas estad\'isticas detalladas por
\citeauthor{monroe2008fightin} y se las utiliza para vectorizar los
discursos emitidos por los legisldores en la vig\'esimo tercera sesi\'on,
en la cual se discuti\'o la ley para el acceso a
la interrupci\'on voluntaria del embarazo.
Se intenta, con esto, evaluar alternativas posibles al tradicional \textit{TF-IDF}
y comparar su desempeño en la selecci\'on de rasgos y entrenamiento de modelos.

Como objetivos generales, este trabajo persique:

\begin{itemize}
    \item{Comparar distintos m\'etodos estad\'isticos que permitan representar
    discursos en t\'erminos num\'ericos.}
    \item{Seleccionar uno o m\'as m\'etodos estad\'isticos que posibiliten
    generar una representaci\'on vectorial de los discursos a favor y en
    contra de la ley de la interrupci\'on voluntaria del embarazo y entrenar
    con ella un modelo predictivo de clasificaci\'on.}
\end{itemize}

Para lograr dichos objetivos, este trabajo se propone, en t\'erminos
particulares:

\begin{itemize}
    \item{Obtener un \textit{corpus} de datos con discursos a favor y en contra
    de la legislaci\'on del aborto sobre el cual se puedan aplicar las t\'ecnicas de
    vecotrizaci\'on y, luego, entrenar un modelo de predicci\'on.}
    \item{En los casos necesarios, implementar los m\'etodos estad\'isticos detallados
    por \cite{monroe2008fightin} en un c\'odigo ejecutable de \textit{python} que
    luego pueda ser aplicado al \textit{corpus} de discursos.}
    \item{Comparar el rendimiento de los distintos m\'etodos de vectorizaci\'on utilizando
    t\'ecnicas propias del aprendizaje autom\'atico, tales como la validaci\'on
    cruzada, y m\'etricas igualmente acordes, como el \textit{accuracy}, la precisi\'on,
    la cobertura y el \textit{score F1}.}
    \item{Entrenar un modelo predictivo de clasificaci\'on, como la Regresi\'on Log\'istica,
    y evaluar su rendimiento al utilizar el m\'etodo de vectoriazi\'on seleccionado.}
\end{itemize}

A continuaci\'on, el lector podr\'a encontrar, en el apartado \ref{section-data},
una descripci\'on sobre la descarga y obtenci\'on de los datos utilizados en este
trabajo, como as\'i tambi\'en de las caracter\'isticas generales que presenta. Las
secciones \ref{section-methods} y \ref{section-results} exponen, respectivamente,
los m\'etodos empleados y los resultados obtenidos a partir de su implementaci\'on.
Ambas poseen la misma estructura: en primer lugar hacen referencia
al proceso de anotaci\'on autom\'atico con supervisi\'on y correcci\'on manual utilizado
para pre-procesar el \textit{corpus}; luego presentan la selecci\'on de rasgos
para vectorizar los discursos y lo referido a la comparaci\'on de estos m\'etodos, y
finalmente se detienen en el entrenamiento de un modelo de clasificaci\'on (la
Regresi\'on Log\'isitca) y su correspondiente evaluaci\'on. La secci\'on
\ref{section-discussion} concluye con algunas reflexiones finales y
discusi\'on al respecto. Hacia el final de este trabajo es posible encontrar el
Anexo, donde se detallan las pautas de anotaci\'on adoptadas y tablas y gr\'aficos
adicionales.
