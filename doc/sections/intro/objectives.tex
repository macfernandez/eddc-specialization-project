En este trabajo y a partir del acceso a los datos de la C\'amara de Senadores,
se retoman las t\'ecnicas estad\'isticas detalladas por
\citeauthor{monroe2008fightin} y se las utiliza para vectorizar los
discursos emitidos por los legisldores en la vig\'esimo tercera sesi\'on,
en la cual se discuti\'o la ley para el acceso a
la interrupci\'on voluntaria del embarazo.
Se intenta, con esto, evaluar alternativas posibles al tradicional \textit{TF-IDF}
y comparar su desempeño en la selecci\'on de rasgos y entrenamiento de modelos.

\subsubsection{Objetivos generales}

Como objetivos generales, este trabajo persigue:

\begin{itemize}
    \item{Comparar distintos m\'etodos estad\'isticos que permitan representar
    discursos en t\'erminos num\'ericos.}
    \item{Seleccionar uno o m\'as m\'etodos estad\'isticos que posibiliten
    generar una representaci\'on vectorial de los discursos a favor y en
    contra de la ley de la interrupci\'on voluntaria del embarazo y entrenar
    con ella un modelo predictivo de clasificaci\'on.}
\end{itemize}

\subsubsection{Objetivos particulares}

En t\'erminos particulares, los objetivos de este trabajo son:

\begin{itemize}
    \item{Obtener un \textit{corpus} de datos con discursos a favor y en contra
    de la legislaci\'on del aborto sobre el cual se puedan aplicar las t\'ecnicas de
    vectorizaci\'on y, luego, entrenar un modelo de predicci\'on.}
    \item{En los casos necesarios, implementar los m\'etodos estad\'isticos detallados
    por \cite{monroe2008fightin} en un c\'odigo ejecutable de \textit{Python} que
    luego pueda ser aplicado al \textit{corpus} de discursos.}
    \item{Entrenar un modelo predictivo de clasificaci\'on, como la regresi\'on log\'istica,
    que utilice distintos m\'etdos de vectorizaci\'on para predecir el voto relacionado
    a cada discurso emitido y, luego, evaluar su rendimiento al utilizar
    el m\'etodo de vectoriazi\'on seleccionado.}
    \item{A partir de utilizar el voto como verdad fundamental a predecir,
    comparar el rendimiento de los distintos m\'etodos de vectorizaci\'on
    a trav\'es de t\'ecnicas propias del aprendizaje autom\'atico, tales como
    la validaci\'on cruzada, y m\'etricas igualmente acordes, como el
    \textit{accuracy}, la precisi\'on, la cobertura y el \textit{score F1}.}
\end{itemize}
