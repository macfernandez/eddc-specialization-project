Inspirado en el estudio de \cite{monroe2008fightin}, este trabajo
perisigue, como objetivos generales:

\begin{itemize}
    \item{Comparar distintos m\'etodos estad\'isticos que permitan representar
    discursos en t\'erminos num\'ericos.}
    \item{Seleccionar uno o más m\'etodos estad\'isticos que permitan
    generar una representación vectorial de los discursos a favor y en
    contra de la ley de la interrupción voluntaria del embarazo y entrenar
    con ella un modelo predictivo de clasificación.}
\end{itemize}

Para lograr dichos objetivos, este trabajo se propone, en términos
particulares:

\begin{itemize}
    \item{Obtener un \textit{corpus} de datos con discursos a favor y en contra
    de la legislación del aborto sobre el cual se puedan aplicar las técnicas de
    vecotrización y, luego, entrenar un modelo de predicción.}
    \item{En los casos necesarios, implementar los métodos estadísticos detallados
    por \cite{monroe2008fightin} en un código ejecutable de \textit{python} que
    luego pueda ser aplicado al \textit{corpus} de discursos.}
    \item{Comparar el rendimiento de los distintos métodos de vectorización utilizando
    métodos propias del aprendizaje automático, tales como la validación
    cruzada, y métricas igualmente acordes, como el \textit{accuracy}, la precisión,
    la cobertura y el \textit{score F1}.}
    \item{Entrenar un modelo predictivo de clasificación, como la Regresión Logística,
    y evaluar su rendimiento al utilizar el método de vectoriazión seleccionado.}
\end{itemize}

A continuación, el lector podrá encontrar, en el apartado \ref{section-data},
una descripción sobre la descarga y obtención de los datos utilizados en este
trabajo, como así también de las características generales que presenta. Las
secciones \ref{section-methods} y \ref{section-results} presentan, respectivamente,
los métodos empleados y los resultados obtenidos a partir de su implementación.
Ambas poseen la misma estructura: en primer lugar hacen referencia
al proceso de anotación automático con supervisión y corrección manual utilizado
para pre-procesar el \textit{corpus}; luego presentan la selección de rasgos
para vectorizar los discursos y lo referido a la comparación de estos métodos, y
finalmente se detienen en el entrenamiento de un modelo de clasificación (la
Regresión Logísitca) y su correspondiente evaluación. La sección
\ref{section-discussion} concluye con algunas reflexiones finales y
discusión al respecto. Hacia el final de este trabajo es posible encontrar el
Anexo, donde se detallan las pautas de anotación adoptadas y algunos gráficos
adicionales.
