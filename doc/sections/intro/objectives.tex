Inspirado en el estudio de \cite{monroe2008fightin}, este trabajo
perisigue, como objetivos generales:

\begin{itemize}
    \item{Analizar distintos m\'etodos estad\'isticos que permitan representar
    discursos en t\'erminos num\'ericos y, as\'i, contrastarlos gr\'aficamente.}
    \item{Utilizar dichos m\'etodos estad\'isticos para realizar una selecci\'on de
    rasgos que permita capturar las particularidades discursivas de los
    votantes a favor y en contra de la despenalizaci\'on del aborto.}
\end{itemize}

En t\'erminos particulares, busca adem\'as:

\begin{itemize}
    \item{Cotejar si las observaciones detalladas en \cite{monroe2008fightin}
    para los m\'etodos estad\'isticos se evidencian
    tambi\'en para el caso del español.}
    \item{Evaluar si los rasgos seleccionados permiten entrenar un modelo que,
    dado un discurso, distinga qu\'e voto podr\'ia llegar a emitir la persona que
    lo pronunci\'o.}
\end{itemize}
