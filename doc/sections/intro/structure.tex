A continuaci\'on, el lector podr\'a encontrar, en el apartado \ref{section-data},
una descripci\'on sobre la descarga y obtenci\'on de los datos utilizados en este
trabajo, como as\'i tambi\'en de las caracter\'isticas generales que presenta. Las
secciones \ref{section-methods} y \ref{section-results} exponen, respectivamente,
los m\'etodos empleados y los resultados obtenidos a partir de su implementaci\'on.
Ambas poseen la misma estructura: en primer lugar hacen referencia
al proceso de anotaci\'on autom\'atico con supervisi\'on y correcci\'on manual utilizado
para pre-procesar el \textit{corpus}; luego presentan la selecci\'on de rasgos
para vectorizar los discursos y lo referido a la comparaci\'on de estos m\'etodos, y
finalmente se detienen en el entrenamiento de un modelo de clasificaci\'on (la
regresi\'on log\'istica) y su correspondiente evaluaci\'on. La secci\'on
\ref{section-discussion} concluye con algunas reflexiones finales y
discusi\'on al respecto. Hacia el final de este trabajo es posible encontrar el
Anexo, donde se detallan las pautas de anotaci\'on adoptadas y tablas y gr\'aficos
adicionales.
