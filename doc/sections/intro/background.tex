El ``Encuentro nacional de mujeres'' consiste en un evento de alcance nacional
en el territorio argentino que se realiza de manera anual desde 1986 y se propone
auto-convocado, democr\'atico, pluralista y federal.
Como fruto de este evento surge, en 2005, la ``Campaña nacional por el derecho al
aborto legal, seguro y gratuito'', una alianza de organizaciones que promueve y
lucha por la educaci\'on sexual como pilar fundamental en la decisi\'on sobre la gestaci\'on,
el acceso a m\'etodos anticonceptions que permitan la adecuada prevenci\'on de embarazos
no deseados, y el aborto legal, seguro y grauito que le posibilite a la persona gestante
discontinuar con un embarazo si no desea atravesarlo.
Hacia 2007, la Campaña redacta y presenta el ``Proyecto de ley de interrupci\'on
voluntaria del embarazo'' como iniciativa social para que sea considerado por los
legisladores y, as\'i, llegue a ser tratado en el Congreso de la Naci\'on.
Pero no es sino hasta junio de 2018, y luego de reiteradas presentaciones, que logran
contar con el aval de entre 20 y 70 diputados. Reci\'en entonces el proyecto llega a
ser debatido en la C\'amara de dichos legisladores, donde logra media sanci\'on.
Posteriormente, la C\'amara de Senadores lo rechaza.
En diciembre de 2020, el proyecto se vuelve a tratar y, esta vez, logra la aprobaci\'on
del Congreso.
La ley 27.610, sobre el ``Acceso a la interrupci\'on voluntaria del embarazo'', entr\'o
en vigencia en la Argentina el 24 de enero de 2021 y regula el acceso a la
interrupci\'on voluntaria del embarazo y a los cuidados postaborto para todas
las personas con capacidad de gestar.\footnote{\citeauthor{campana@lalucha}.}
\footnote{\citeauthor{huesped@historia}.}
\par
Por otra parte, la ley 27.275 garantiza el acceso a la informaci\'on p\'ublica y permite su b\'usqueda,
acceso y solicitud, como as\'i tambi\'en su an\'alisis, procesamiento, uso y distribuci\'on.
Se considera informaci\'on p\'ublica a todos aquellos datos generados, obtenidos o
controlados por los organismos del Estado y empresas indicados en la misma ley, entre
los cuales se encuentra el Poder Legislativo de la
Naci\'on\footnote{\citeauthor{minjusticia@accesoinfo}}, conformado por la C\'amara
de Diputados y Senadores.
Luego de sus sesiones, ambas c\'amaras disponibilizan en formato digital la
transcripci\'on taquigr\'afica de la reuni\'on, de modo que cualquier ciudadano con
acceso a una computadora e internet pueda acceder a ella.
\par
Al trabajar con textos, el cient\'ifico de datos se ve en la necesidad
de vectorizar, mediante alg\'un mecanismo, el \textit{corpus} que desea
utilizar.
De este modo, lo convierte en una secuencia de n\'umeros capaz de ser
procesada por una computadora siguiendo cierto procedimiento.
Los textos, una vez transformados en vectores, pueden ser agrupados, 
analizados para la obtenci\'on de informaci\'on o utilizados
como conjunto de entrenamiento y testeo de modelos predictivos.
En el cap\'itulo 6 de \usebibentry{jurafsky2000speech}{title}, Jurafsky
repasa los distintos usos y ventajas de la vectorizaci\'on de documentos, a la vez
que detalla los procedimientos m\'as utilizados en aprendizaje autom\'atico. Entre ellos,
menciona el c\'alculo de \textit{TF-IDF} como uno de los m\'etodos \textit{baselines}.
\par
Por su parte, \cite{monroe2008fightin} exponen una serie de t\'ecnicas
que podr\'ian ser \'utiles a la hora de capturar palabras con contenido partidario
en el discurso pol\'itico y los aplican sobre datos tomados del Senado de los Estados
Unidos. Sus objectivos se centran en la selecci\'on y evaluaci\'on de rasgos, a trav\'es
de los cuales esperan distinguir qu\'e palabras indican la adopci\'on de una u otra postura
y, de ah\'i, que puedan usarse como \textit{features} confiables a la hora de entrenar un
modelo. A su vez, respecto de la evaluaci\'on, pretenden no solo comprender qu\'e
palabras caracterizan a determinado partido sino, adem\'as, en qu\'e medida o grado
lo caractirzan, cu\'ales son las palabras m\'as partidarias o representativas de ciertas
posturas.
