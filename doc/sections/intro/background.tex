El ``Encuentro nacional de mujeres'' consiste en un evento de alcance nacional
en el territorio argentino que se realiza de manera anual desde 1986, y se propone
auto-convocado, democrático, pluralista y federal.
Como fruto de este evento surge, en 2005, la ``Campaña nacional por el derecho al
aborto legal, seguro y gratuito'', una alianza de organizaciones que promueve y
lucha por la educación sexual como pilar fundamental en la decisión sobre la gestación,
el acceso a métodos anticnceptions que permitan la adecuada prevención de embarazos
no deseados, y el aborto legal, seguro y grauito que le posibilite a la persona gestante
discontinuar con un embarazo si no desea atravesarlo.
Hacia 2007, la Campaña redacta y presenta el ``Proyecto de ley de interrupción
voluntaria del embarazo'' como iniciativa social para que sea considerado por los
legisladores y, así, llegue a ser tratado en el Congreso de la Nación.
Pero no es sino hasta junio de 2018, y luego de reiteradas presentaciones que contaron
con el aval de entre 20 y 70 diputados, que el proyecto llega a ser debatido en en la
Cámara de dichos legisladores, donde logra media sanción.
Posteriormente, la Cámara de Senadores rechaza su aprobación.
En diciembre de 2021, el proyecto se vuelve a tratar y, esta vez, logra la aprobación
del Congreso.
La ley 27.610, sobre el ``Acceso a la interrupción voluntaria del embarazo'', entró
en vigencia en la Argentina el 24 de enero de 2021, y regula el acceso a la
interrupción voluntaria del embarazo y a los cuidados postaborto a todas
las personas con capacidad de gestar.\footnote{\citeauthor{campana@lalucha}.}
\footnote{\citeauthor{huesped@historia}.}
\par
Por otra parte, la ley 27.275 garantiza el acceso a la información pública y permite su búsqueda,
acceso y solicitud, como así también su análisis, procesamiento, uso y distribución.
Se considera información pública a todos aquellos datos generados, obtenidos o
controlados por los organismos del Estado y empresas indicados en la misma ley, entre
los cuales se encuentra el Poder Legislativo de la
Nación\footnote{\citeauthor{minjusticia@accesoinfo}}, conformado por la Cámara
de Diputados y Senadores.
Luego de sus sesiones, ambas cámaras disponibilizan, en formato digital, la
transcripción taquigráfica de la reunión, de modo que cualquier ciudadano con
acceso a una computadora e internet pueda acceder a ella.
\par
Al trabajar con textos, el científico de datos se ve en la necesidad
de vectorizar, mediante algún mecanismo, el \textit{corpus} con el que desea
trabajar.
De este modo, lo convierte en una secuencia de números capaz de ser
procesada por una computadora siguiendo cierto procedimiento.
Los textos, una vez convertidos en vectores, pueden ser agrupados, utilizados
como conjunto de entrenamiento y testeo de modelos predictivos, o analizados para
la obtención de información.
En el capítulo 6 de \textit{Speech and language processing}, Jurafsky
repasa los distintos usos y ventajas de la vectorización de documentos, a la vez
que detalla los procedimientos más utilizados en aprendizaje automático. Entre ellos,
menciona el cálculo de \textit{TF-IDF} como uno de los métodos \textit{baselines}.
\par
Por su parte, \cite{monroe2008fightin} exponen una serie de técnicas
que podrían ser útiles a la hora de capturar palabras con contenido partidario
en el discurso político, y los aplican sobre datos tomados del Senado de los Estados
Unidos. Sus objectivos se centran en la selección y evaluación de rasgos, a través
de los cuales esperan distinguir qué palabras indican la adopción de una u otra postura
y, de ahí, que pueda usarse como un \textit{feature} confiable a la hora de entrenar un
modelo. A su vez, respecto de la evaluación, pretenden no solo comprender qué
palabras caracterizan a determinado partido sino, además, en qué medida o grado
lo caractirzan, cuáles son las palabras más partidarias o representativas de ciertas
posturas.
