El ``Encuentro nacional de mujeres'' consiste en un evento de alcance nacional
en el territorio argentino que se realiza de manera anual desde 1986 y se propone
auto-convocado, democr\'atico, pluralista y federal.
Como fruto de este evento surge, en 2005, la ``Campaña nacional por el derecho al
aborto legal, seguro y gratuito'', una alianza de organizaciones que promueve y
lucha por la educaci\'on sexual como pilar fundamental en la decisi\'on sobre la gestaci\'on,
el acceso a m\'etodos anticonceptions que permitan la adecuada prevenci\'on de embarazos
no deseados, y el aborto legal, seguro y grauito que le posibilite a la persona gestante
discontinuar con un embarazo si no desea atravesarlo.
Hacia 2007, la Campaña redacta y presenta el ``Proyecto de ley de interrupci\'on
voluntaria del embarazo'' como iniciativa social para que sea considerado por los
legisladores y, as\'i, llegue a ser tratado en el Congreso de la Naci\'on.
Pero no es sino hasta junio de 2018, y luego de reiteradas presentaciones, que logran
contar con el aval de entre 20 y 70 diputados. Reci\'en entonces el proyecto llega a
ser debatido en la C\'amara de dichos legisladores, donde logra media sanci\'on.
Posteriormente, la C\'amara de Senadores lo rechaza.
En diciembre de 2020, el proyecto se vuelve a tratar y, esta vez, logra la aprobaci\'on
del Congreso.
La ley 27.610, sobre el ``Acceso a la interrupci\'on voluntaria del embarazo'', entr\'o
en vigencia en la Argentina el 24 de enero de 2021 y regula el acceso a la
interrupci\'on voluntaria del embarazo y a los cuidados postaborto para todas
las personas con capacidad de gestar.\footnote{\cite{campana@lalucha}.}
\footnote{\cite{huesped@historia}.}
\par
Por otra parte, en su publicaci\'on \usebibentry{minjusticia@accesoinfo}{title},
el \cite{minjusticia@accesoinfo} difunde que la ley 27.275 garantiza el acceso a la
informaci\'on p\'ublica y permite su b\'usqueda,
acceso y solicitud, como as\'i tambi\'en su an\'alisis, procesamiento, uso y
distribuci\'on.
Se considera informaci\'on p\'ublica a todos aquellos ``datos que generan, obtienen,
transforman, controlan o cuidan los organismos del Estado y empresas indicados en la ley
''. Estos datos, a su vez, ``tienen que estar contenidos en documentos de cualquier
formato o soporte: pueden estar en papel, en archivos digitales, etc.''. Entre los
organismos del Estado mencionados, se encuentra el Poder Legislativo de la
Naci\'on, conformado por la C\'amara de Diputados y Senadores, los cuales
Luego de sus sesiones, ambas c\'amaras disponibilizan en formato digital la
transcripci\'on taquigr\'afica de la reuni\'on, de modo que cualquier ciudadano con
acceso a una computadora e internet pueda acceder a ella.
\par
\cite{jurafsky2000speech}\footnote{
Las citas incluidas a continuaci\'on fueron traducidas del ingl\'es por
la autora de este trabajo.}
reseña una serie de aspectos relacionados con la semántica léxica, la cual se ocupa
del estudio lingü\'istico del significado de las palabras, para argumentar que
``un modelo del significado de las palabras nos debería permitir realizar inferencias
para abordar tareas relacionadas al significado''. Entre dichos aspectos, menciona
la noción de similitud. De ella dice que ``mientras que las palabras no tienen muchos
sin\'onimos, la mayoría de ellas s\'i tiene muchas palabras similares. [...]
Al movernos de la sinonimia a la similitud, es \'util pasar de hablar de relaciones
entre sentidos de las palabras (como en la sinonimia) a hablar de relaciones entre las
palabras (como en la similitud). Lidiar con palabras nos evita tener que comprometernos
con una representaci\'on particular del sentido de las palabras, lo cual simplifica
nuestra tarea. [...] Conocer cuán similares son dos palabras puede ayudar a computar
cuán similar es el significado de dos frases u oraciones [...]''. Del mismo modo
se detiene en otro tipo de relaci\'on entre las palabras como es el campo
sem\'antico, el cual ``refiere a un conjunto de palabras que cubren un dominio
sem\'antico particular y mantienen relaciones estructurales entre s\'i'', y también en
la connotación o el sentio afectivo de las palabras, el cual indica ``los
aspectos del significado de una palabra que est\'an relacionados con las emociones,
sentimientos, opiniones y evaluaciones de un escritor o lector''. En particular,
el an\'alisis de sentimientos permite identificar ``el lenguaje usado para evaluaciones
positivas o negativas''. Luego de enunciar estas y otras cuestiones de interés para
el an\'alisis sem\'antico, Jurafsky resalta que la ``la semántica vectoial es el método
tradicional de repressentar el significado de las palabras en el PLN\footnote{Nota de
traducci\'on: la sigla PLN refiere a Procesamiento de Lenguaje Natural.}, ayud\'andonos
a modelar muchos de los aspectos del significado de las palabras'', y menciona que
``en el modelo de \textit{TF-IDF}, un modelo base de importante, el significado
de una palabra es definifo por una simple funci\'on de c\'alculo de las palabras
cercanas [...] este m\'etodo resulta en un vectores muy extendos que adem\'as
son ralos\footnote{Nota de traducci\'on: se traduce como ralo el t\'ermino del ingl\'es
\textit{sparse}.}''.
\par
As\'i, es la vectorizaci\'on la que le permite al cient\'ifico de datos convertir,
mediante alg\'un c\'alculo pre-establecido, el \textit{corpus} que desea
utilizar en una secuencia de n\'umeros capaz de ser
procesada por una computadora siguiendo cierto procedimiento.
Los textos, transformados en vectores, pueden ser agrupados,
analizados para la obtenci\'on de informaci\'on o utilizados
como conjunto de entrenamiento y testeo de modelos predictivos.
\par
Por su parte, \cite{monroe2008fightin} exponen una serie de t\'ecnicas
que podr\'ian ser \'utiles a la hora de capturar palabras con contenido partidario
en el discurso pol\'itico y los aplican sobre datos discursivos tomados del Senado de
los Estados Unidos, disponibilizados de manera pública y en formato digital.
Sus objetivos se centran en la selecci\'on y evaluaci\'on de rasgos, a trav\'es
de los cuales esperan distinguir qu\'e palabras indican la adopci\'on de una u otra postura
y, de ah\'i, que puedan usarse como \textit{features} confiables a la hora de entrenar un
modelo. A su vez, respecto de la evaluaci\'on, pretenden no solo comprender qu\'e
palabras caracterizan a determinado partido sino, adem\'as, en qu\'e medida o grado
lo caractirzan, cu\'ales son las palabras m\'as partidarias o representativas de ciertas
posturas.
