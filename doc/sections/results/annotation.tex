Luego del proceso de anotación y corrección de lemas, se extrajeron algunas métricas
de resumen y se realizaron gráficos con el fin de comprender qué distribución
presentaban las etiqutas utilizadas y cuántos de los datos etiquetdados de forma
automática requirieron alguna corrección, ya fuera en el lema asignado, ya en
la etiqueta gramatical predicha. Los criterios de corrección utilizados pueden
encontrarse en el apéndice \ref{appendix-annotation}.
\par
En total, se encontró un conjunto de 8324 \textit{tokens} únicos, donde la unicidad no
hace referencia solamente a la secuencia de caracteres sino también a la etiqueta
gramatical asignada. En contraste con la figura \ref{fig-distrib-tokens}, donde
la unicidad implica igualdad en dicha secuencia y, entonces, dos palabras que
contienen los mismos caracteres en el mismo orden son consideradas el mismo
\textit{token}, aquí también se tiene en cuenta que, además, reciben el mismo
\textit{POS tag}. Por ejemplo: en el análisis exploratorio realizado
para la descripción de los datos, el \textit{token} `suma' se considera una sola vez,
independientemente de si, en sus distintas ocurrencias, refiere al verbo
(en la tercera persona singular del verbo `sumar') o al nombre
(\textit{Acción y efecto de sumar}\footnote{\Citeauthor{rae23diccionario}}). En
este apartado, en cambio, al agrupar los \textit{tokens} por caracters y etiqueta
gramatical, estas dos ocurrencias son consideradas por separado.
\par
