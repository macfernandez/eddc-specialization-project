Para este trabajo se utilizó la versión taquigráfica de la sesión número
23 (reunión 28) celebrada por la Cámara de Senadores, en la que se abordó la
regulación del acceso a la interrupción voluntaria del embarazo y a sus
posteriores cuidados. Los datos fueron obtenidos de la página del Senado de la Nación
Argentina\footnote{https://www.senado.gob.ar/parlamentario/sesiones/29-12-2020/28/}
utilizando el método de \textit{scrapping}.\par
En primer lugar, se descargó la transcripción en formato \textit{PDF} y, luego, por
medio de la librería \textit{pdfminer}\footnote{https://pdfminersix.readthedocs.io/en/latest/},
se lo convirtió a texto plano, de manera que fuese procesable. Dado que la librería
empleada en la conversión no arrojó una conversión limpia y que, además, no todo el texto
resultaba de relevancia para el presente trabajo, se realizó una tarea de
preprocesamiento, con el fin de limpiar y organizar los datos de forma útil a los
objetivos aquí perseguidos.\par
Como primer paso del preprocesamiento se quitaron los encabezados y pie de paǵina.
El archivo \textit{.pdf}, convertido a \textit{.txt}, no contaba con separación de
páginas, por lo que los encabezados y pies se convertían en líneas de texto que
interrumpían los discursos de los senadores y participantes del debate.
Posteriormente, se extrajo la sección del texto pertinente para este trabajo:
la sección 6, dedicada a la ``Regulación  del  acceso  a  la  interrupción
voluntaria  del  embarazo  y  a  la atención postaborto''. Otras secciones consistían
en el izamineto de la bandera, la convocatoria a la sesión, la lectura de la ley
resultante, entre otras cuestiones que no hacían al discurso argumentativos de los
asistentes, sino más bien a cuestiones protocolares, por lo que fueron desestimadas
para este análisis.\par
Con el texto de interés delimitado, y mediante el uso de patrones regulares, se
identificó a los distintos oradores y sus respectivos discursos, como así también
a las secciones que no pertenecían a fragmentos discursivos emitidos durante la
discusión, sino a comentarios agregados por el taquígrafo\footnote{``Luego de unos
instantes'' y ``Contenido no inteligible'' constituyen ejemplos de estos
fragmentos.}. Siguiendo la metodología descripta por \cite{monroe2008fightin},
los datos fueron guardados en un archivo en formato \textit{XML}, en el que
cada fragmento fue etiquetado con las siguiente información:
\begin{itemize}
    \item \texttt{speech}: etiqueta que indica si el fragmento corresponde a un
    discurso efectivamente emitido durante la discusión, en cuyo caso toma el
    valor \texttt{true}, o si consiste en un comentario del taquígrafo, ante lo cual
    toma el valor \texttt{false}.
    \item \texttt{speaker}: en el caso de que el contenido del texto refiera a
    un fragmento discursivo, esta etiqueta indica el nombre del orador que lo
    pronunció. Si el fragmento es un comentario del traquígrafo, la etiqueta
    recibe el valor \texttt{none}.
\end{itemize}
\par
Adicionalmente, como resultado del preprocesamiento, se extrajo la lista de
asistentes a la sesión en formato tabular, con el objetivo de poder
enriquecerla adicionando especificaciónes como la filiación partidaria y la
decisión de voto.\todo[color=blue!40]{Desarrollar y explicar el scrapping y mergeo de la data.}

% =================================================
% Breve descripción de los datos:
% - cantidad de partidos totales
% - cantidad de oradores totales
% - distribución oradores~partidos
% - distribución género~partidos
% - palabras y tokens totales
% - distribución de palabras y tokens por partido
% - distribución de palabras y tokens por orador