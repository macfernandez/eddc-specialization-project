Una vez elegido el vectorizador a utilizar (\textit{TF-IDF} tomando
logaritmo de la frecuencia inversa en documentos), se lo emple\'o para
vectorizar el \textit{corpus} y, con los datos representados
num\'ericamente, se realizo una selecci\'on
de hiperpar\'ametros para entrenar el modelo de regresi\'on log\'istica\footnote{
En este procesamiento solamente se emple\'o el vectorizador elegido.}.
Utilizando el m\'etodo de b\'usqueda exhaustiva, se
exploraron distintas combinaciones que afectan a
al peso asignado a las clases a predecir y la regularizaci\'on,
un procedimiento que permite penalizar el aprendizaje de modo tal que
el modelo no resulte sobreajustado a los datos de entrenamiento y ello
dificulte las correctas predicciones sobre datos no vistos. Las
alternativas evaluadas consisten en:

\begin{itemize}
    \item Peso de las clases (\textit{class\_weight}): sin peso; pesos
    inversamente proporcionales a los observados en los datos de entrenamiento;
    pesos fijos calculados a partir de los datos de ese mismo \textit{dataset}.    
    \item Regularizaci\'on (\textit{penalty}): sin regularizaci\'on;
    regularizaci\'on \textit{L1}
    o de \textit{Lasso}; regularizaci\'on \textit{L2} o de \textit{Ridge}.
    \item Intensidad inversa de la regularizaci\'on (\textit{C})\footnote{
    La implementaci\'on de \textit{scikit-learn} para la regresi\'on
    log\'istica no emplea el par\'ametro $\lambda$ sino el par\'ametro \textit{C},
    el cual refiere a la intensidad inversa de la regularizaci\'on y puede
    calcularese mediante la f\'ormula $\frac{1}{C}$, por lo que valores
    m\'as pequeños conllevan a una regularizaci\'on m\'as fuerte y valores m\'as
    grandes, a una m\'as d\'ebil.}:
     $0.1$; $0.5$; $2$; $1$.
\end{itemize}

Durante la b\'usqueda exhaustiva se realiz\'o una validaci\'on cruzada del mismo
modo que al seleccionar el vectorizador: colocando una semilla para que los
resultados fuesen replicables, con una segmentaci\'on del conjunto de datos
estratificada seg\'un la variables objetivo y con cinco \textit{folds} o
iteraciones.
\par
Luego, con los mejores hiperpar\'ametros hallados, se entren\'o un nuevo modelo
con el conjunto de entrenamiento completo, reservando solamente el
conjunto de testeo separado al principio del trabajo
para realizar la evaluaci\'on final.
