Siguiendo a \cite{monroe2008fightin}, se aplic\'o una serie de t\'ecnicas
no basadas en modelos probabil\'isticos para la selecci\'on de rasgos.
Dos de los m\'etodos aplicados (\textit{TF-IDF} y \textit{Word Scores}) consisten
en algoritmos para la asignaci\'on de puntuaciones o \textit{scores} a las
palabras dentro de un \textit{corpus}. El resto de las t\'ecnicas utilizan
m\'etricas estad\'isticas sencillas. Para la aplicación de estos procedimientos
se utilizaron solamente las lematizaciones de las palabras, sin su categoría
gramatical.

\subsubsection{Diferencia de frecuencias absolutas}
En primer lugar, se contrastaron los lemas de los discursos de los
senadores con votos positivos conlos de los de senadores con votos negativos
utilizando la diferencia de frecuencias absolutas:

\begin{equation}
y_w = y_{w}^{(P)}-y_{w}^{(N)}
\end{equation}

donde $y_{w}^{(X)}$ refiere a la cantidad de ocurrencias absolutas del lema
\textit{w} en los discursos que votaron de manera positiva (P) y negativa (N).
En los casos en los que $y_{w}$ sea mayor o igual a $0$, el lema \textit{w}
ser\'a caracter\'istico del discurso de los senadores que votaron a favor de
la legalizaci\'on. En caso contrario, ser\'a caracter\'istica de quienes votaron
en contra.

\subsubsection{Diferencia de proporciones}
El segundo contraste realizado utiliz\'o, en lugar de las frecuencias absolutas,
la proporci\'on de los lemas en cada conjunto de discursos.

\begin{equation}
    f_w = f_{w}^{(P)}-f_{w}^{(N)}
\end{equation}


donde $f_{w}^{(X)}$ se define como $y_{w}^{(X)} / n^{(X)}$, siendo $y_{w}^{(X)}$
la frecuencia absoluta del lema $w$ en el cojunto los discursos de $X$
y $n^{(X)}$, la cantidad de lemas totales en esos discursos.

El mismo abordaje se intent\'o removiendo previamente las palabras consideradas
\textit{stopwords}. Para esto se probaron dos abordajes. Por un lado, se
utiliz\'o el conjunto de palabras predefinidas como tales de la librer\'ia \textit{NLTK}
\footnote{\url{https://raw.githubusercontent.com/nltk/nltk_data/gh-pages/packages/corpora/stopwords.zip}}
y, por otro, se definió un conjunto de palabras consideradas \textit{stopwords}
en el universo de estos discursos a partir de aplicar un análisis utilizando la
Ley de Zipf.\todo[color=pink]{Expicar qué es la Ley de Zipf}

\subsubsection{Ratio de \textit{Odds}}
Asimismo, se compararon los \textit{odds} de ambos conjuntos de discursos:

\begin{equation}
    O_w = \frac{O_{w}^{(P)}}{O_{w}^{(N)}}
\end{equation}

aqu\'i, $O_{w}^{(X)}$ consiste en $f_{w}^{(X)}/(1-f_{w}^{(X)})$.

\subsubsection{Ratio \textit{Log-odds}}
Sobre los \textit{odds}, tambi\'en se calcul\'o el \textit{ratio log-odds} y
se realiz\'o una comparaci\'on con ellos. Para este c\'alculo se utiliz\'o el
logaritmo natural: $\ln{O_w}$

\subsubsection{\textit{TF-IDF}}
Los autores del art\'iculo ``\usebibentry{monroe2008fightin}{title}'' indican
que utilizaron tambi\'en, como estad\'istica comparativa, el algoritmo de asignaci\'on
de peso \textit{TF-IDF}. Siguiendo su proceder, en este trabajo se intent\'o
hacer lo mismo. Para ello, fue necesario implementar el c\'alculo, dado que
la f\'ormula utilizada en el \textit{paper} no era la misma que la implementada
por \textit{scikit-learn}, la librer\'ia usual para extraer
este tipo de \textit{scores}.\footnote{Para m\'as informaci\'on sobre las f\'ormulas
implementadas por \textit{scikit-learn}, visitar:
\url{https://scikit-learn.org/stable/modules/feature_extraction.html\#tfidf-term-weighting}}\par
Los autores indican haber usado dos fórmulas diferentes: ambar utilizando la frecuencia
de término natural y sin normalizar, pero una con la frecuencia de documentos
con logaritmo (fórmula \ref{equation-tfidf-ntn}) y la otra, con frecuencia de documento
natural (fórmula \ref{equation-tfidf-nnn}):

\begin{equation}
\label{equation-tfidf-ntn}
    tf.idf_{w}^{(X)}(ntn) = f_{w}^{(X)} \times \ln\bigg({\frac{1}{df_{w}}}\bigg)
\end{equation}

\begin{equation}
\label{equation-tfidf-nnn}
    tf.idf_{w}^{(X)}(nnn) = \frac{f_{w}^{(X)}}{df_{w}}
\end{equation}

donde $f_{w}^{X}$ refiere a la proporci\'on del lema $w$ en el discurso
perteneciente a $X$ (con $X \in \lbrace P,N \rbrace$) y $df_w$ refiere a la
cantidad de documentos totales en los que aparece el lema $w$.
\par
Lo que los autores no mencionan es qu\'e tipo de comparaci\'on realizan entre los
lemas pertenecientes a los distintos discursos. Siguiendo los procedimientos
anteriores, en este trabajo se opt\'o por usar la sustracci\'on, por lo que:
\todo[color=pink]{en resultados, recordar hablar de lo pol\'emico de este abordaje
para}

\begin{equation*}
    tf.idf_{w} = tf.idf_{w}^{P}-tf.idf_{w}^{N}
\end{equation*}

\subsubsection{\textit{Word Scores}}
Por \'ultimo, se utiliz\'o tambi\'en el procedimiento \textit{WordScores} descripto por
\cite{laver2003extracting}, que le asigna un peso a los lemas a partir
del siguiente c\'alculo:\todo[color=pink]{checkear si es de ellos o de qui\'en}

\begin{equation*}
    W_{w}^{^*(P-N)} = \frac{y_{w}^{(P)}/n^{(P)}-y_{w}^{(N)}/n^{(N)}}{y_{w}^{(P)}/n^{(P)}+y_{w}^{(N)}/n^{(N)}}n_{w}
\end{equation*}\todo[color=pink]{leer lowe 2008, understanding wordscores y demostar
this is nearly identical to the difference of proportions measure (correlate at over +0.998)}
