Dado que la intención de este trabajo consiste en utilizar los datos textuales
obtenidos de la sesión del Congreso en cuestión para realizar una selección
de rasgos a fin de entrenar distintos modelos, resulta crucial procesar dichos
datos de forma tal que el proceso de extracción de información por parte
de un algoritmo de aprendizaje automático se vea faciliado. En concreto, palabras
como `aceptaron', `acepta', `aceptó' refieren las tres a la misma acción (`aceptar'),
pero a la hora de extraer los rasgos a partir de los discursos prounciados serán
consideradas como tres rasgos distintos si no las convertimos a una forma única o
base previamente.
\par
El \textbf{lema} es la forma de una palabra que se utiliza para hacer referencia a ella
en la entrada de un diccionario y varía según la convención en cada lengua. En español
y en inglés, por ejemplo, los verbos se enuncian en infinitivo, pero en latín lo hacen
utilizando la primera persona del singular del presente en modo indicativo, la segunda
del mismo tiempo y la primera del pretérito perfecto. Por otro lado, el
\textbf{\textit{stemizado}} es el proceso por el cual se le remueven los sufijos
a una palabra, persistiendo solamente su primer o primeros morfemas y utilizando
\todo[color=pink]{Lematizado y stemmizado: consultar con FerCa}
esto como forma base\footnote{\citet[Capítulo~3]{bird2009natural}}.
\par
En este trabajo se exploaron ambas opciones de procesamiento. Cuestiones que se tuvieron
en cuenta particularmente fueron la preservación de la marcación de género, puesto que
no resulta menor respecto del tema en consideración, y la minimización de formas a las
cuales se transformaron los verbos. Para el \textit{stemizado} se utilizó la
libería NLTK, la cual proporciona distintos algoitmos posibles. Aquí, se optó por la
implementación de \textit{SnowballStemmer}\footnote{Sobre la implementación de la
librería, consultar la
\href{https://www.nltk.org/api/nltk.stem.SnowballStemmer.html}{documentación existente}.
Y, para más iformación sobre el algoritmo en español, dirigirse a},que permite elegir la lengua sobre la cual
se quiere trabajar.

Si bien en este trabajo se exploraron ambas opciones de procesamiento, dado que el
español presenta una considerable variedad de verbos irregulares
\todo[color=pink]{Morfología irregularidad: pedir referecia a FerCa}, se prefirió la
opción del lematizado por permitir convertir estas ocurrencias a un única forma.
El \textit{stemizado}, en cambio, realizó más de una conversión posible. Verbos
como `aprobar', por ejemplo, fueron transformados en `aprob', para los casos como
`aprobamos' o `aprobaron' y `aprueb', para aquellos como `aprueba', `apruebe'.
\par