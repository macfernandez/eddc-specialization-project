Dado que la intenci\'on de este trabajo consiste en utilizar los datos textuales
obtenidos de la sesi\'on del Congreso en cuesti\'on para realizar una selecci\'on
de rasgos a fin de entrenar un modelo de clasificaci\'on, resulta crucial procesar dichos
datos de forma tal que el proceso de extracci\'on de informaci\'on por parte
de un algoritmo de aprendizaje autom\'atico se vea faciliado. En concreto, palabras
como `aceptaron', `acepta', `acept\'o' se vinculan gramatical y sem\'anticamente:
las tres pertenecen a la categor\'ia de los verbos y refieren a la misma acci\'on (`aceptar').
Sin embargo, a la hora de extraer los rasgos a partir de los discursos prounciados,
estas palabras ser\'an consideradas como tres rasgos distintos si no se las convierte
previamente a una forma \'unica o base.
\par
El \textbf{lema} es la forma de una palabra que se utiliza para hacer referencia a ella
en la entrada de un diccionario y var\'ia seg\'un la convenci\'on en cada lengua. En español
y en ingl\'es, por ejemplo, los verbos se enuncian en infinitivo, pero en lat\'in lo hacen
utilizando la primera persona del singular del presente en modo indicativo, la segunda
del mismo tiempo y la primera del pret\'erito perfecto. Por otro lado, el
\textbf{\textit{stemizado}} es el proceso por el cual se le remueven los sufijos
a una palabra, persistiendo solamente su primer o primeros morfemas y utilizando
esto como forma base\footnote{\citet[Cap\'itulo~3]{bird2009natural}.}.
\par
En este trabajo se exploraron ambas opciones de procesamiento. Cuestiones que se tuvieron
en cuenta particularmente fueron la preservaci\'on de la marcaci\'on de g\'enero, puesto que
no resulta menor respecto del tema en consideraci\'on, y la minimizaci\'on en la cantidad de formas a las
cuales se transformaron los verbos. Para el \textit{stemizado} se utiliz\'o la
liber\'ia \textit{NLTK}\footnote{\url{https://www.nltk.org/} [\'ultimo acceso: 15--10--2024].},
la cual proporciona distintos algoritmos posibles. Aqu\'i, se opt\'o por la
implementaci\'on de \textit{SnowballStemmer}\footnote{Sobre la implementaci\'on de la
librer\'ia, consultar la documentaci\'on en
\url{https://www.nltk.org/api/nltk.stem.SnowballStemmer.html}
[\'ultimo acceso: 15--10--2024]. Y, para m\'as informaci\'on
sobre el algoritmo en español, dirigirse a
\url{https://snowballstem.org/algorithms/spanish/stemmer.html}
[\'ultimo acceso: 15--10--2024].}, que permite elegir
la lengua sobre la cual se quiere trabajar. Respecto del lematizado, se recurri\'o
a la librer\'ia \textit{spaCy}\footnote{\url{https://spacy.io/}
[\'ultimo acceso: 15--10--2024].}, que cuenta con modelos
pre-entrenados para la predicci\'on de lemas en distintos idiomas. El modelo aqu\'i
utilizado es el \textit{es-core-news-md}, entrenado con textos
provenientes de noticias y medios de
comunicaci\'on\footnote{\url{https://spacy.io/models/es\#es_core_news_md}
[\'ultimo acceso: 15--10--2024].}.
\par
En lo que refiere a la marcaci\'on de g\'enero, ninguno de los procedimientos se mostr\'o
superior por sobre el otro para los prop\'ositos de este trabajo, dado que ambos
remueven la marcaci\'on de este accidente, ya sea eliminando el sufijo que lo
representa (`-o' y `-a' en la mayor\'ia de los casos) como lo hace el
\textit{stemizado}, ya transformando la palabra completa a su forma can\'onica
(en masculino singular, en el español) como lo hace el lematizado. En el caso
de la conversi\'on de los verbos a una forma base, en cambio, el lematizado permiti\'o
una mejor reducci\'on, dado que posibilit\'o transformar las distintas ocurrencias
de un verbo a una \'unica forma: el infinitivo. Al ser un procedimiento que se
centra \'unicamente en la remoci\'on de los sufijos, el \textit{stemizado} no permite
lo mismo en el caso de una lengua como el español, que presenta una considerable
variedad de verbos irregulares al nivel de la ra\'iz: tras el
\textit{stemizado}, las distintas ocurrencias de un mismo verbo resultan reducidas
a tantas formas como irregularidades presente su ra\'iz. Ejemplo de esto es el
verbo `aprobar', que fue transformado en `aprob', para los casos como
`aprobamos' o `aprobaron' y `aprueb', para aquellos como `aprueba',
`apruebe'. Es por esto que, si bien ambos procesamientos requirieron una revisi\'on
y correcci\'omn ulteriores, se prefiri\'o el lematizado por sobre el \textit{stemizado}
dado que permiti\'o una transformaci\'on de los verbos m\'as clara para el español.
\par
En concreto, el flujo de trabajo consisti\'o en procesar cada discurso emitido
con los modelos pre-entrenados de la librer\'ia \textit{spaCy}, los cuales permitieron
predecir, para cada \textit{token}, su lema y \textit{POS tag}\footnote{El
\textit{POS tag}, por \textit{Part of Speech}, o etiquetado gramatical consiste
en la asignaci\'on a cada palabra de su categor\'ia gramatical o, en algunos casos,
sint\'actica. Los modelos de la
librer\'ia \textit{spaCy} emplean el sistema de etiquetas de \textit{Universal Dependencies}.
Para m\'as informaci\'on, visitar
\url{https://spacy.io/usage/linguistic-features\#pos-tagging}
[\'ultimo acceso: 15--10--2024]
y \url{https://universaldependencies.org/u/pos/}
[\'ultimo acceso: 15--10--2024].}. Luego, esta
informaci\'on fue volcada en una tabla con tres columnas: una para las ocurrencias
reales de las palabras, otra para los lemas asignados y una tercera con las etiquetas predichas. En un paso siguiente, estas asignaciones fueron corregidas
manualemente, y los lemas y etiquetas obtenidos fueron utilizados para la
extracci\'on de rasgos en el entrenamiento de modelos que se detalla a continuaci\'on.
Los criterios de correcci\'on utilizados pueden encontrarse en el Anexo
\ref{appendix-annotation}, y la secci\'on \ref{subsec-results-annotation}
proporciona un breve an\'alisis de los resultados de este proceso.
