Como parte de la vida pol\'itica de una naci\'on, los ciudadanos y
agrupaciones sociales tienen la posibilidad de presentar proyectos
de ley ante los legisladores. En caso de contar con un aval suficiente,
estos proyectos pueden ser abordados en las c\'amaras de diputados
y senadores y llegar a constituirse como ley efectivamente.
Acorde a lo estipulado en la ley 27.275, que garantiza el acceso a la
informaci\'on p\'ublica, las sesiones de la C\'amara de Sendores son
transcriptas y disponibilizadas en la p\'agina del Senado de la Naci\'on.
\par
En el trabajo llevado a cabo, se utiliz\'o el lenguaje de
programaci\'on \textit{Python 3} para acceder, mediante la t\'ecnica
\textit{scraping}, a los datos proporcionados por el Estado. Se los ha
pre-procesado de manera que la informaci\'on proveniente de
distintas fuentes pudiese ser relacionada, y se los ha etiquetado
siguiendo un procedimiento h\'ibrido que comprendi\'o el uso de modelos
pre-entrnados por librer\'ias de c\'odigo abierto y la correci\'on y
anotaci\'on manual. Estos datos fueron utilizados luego para entrenar modelos
de regresi\'on log\'istica con distintos vectorizadores que se desarrollaron
para para este trabajo y cuya implementaci\'on estuvo basada en aquellos
reseñados por \cite{monroe2008fightin}. Dichos vectorizadores fueron
contrastados respecto de su rendimiento y el mejor fue utilizado para entrenar
un modelo final de regresi\'on log\'istica, para el que previamente fueron
seleccionados los mejores hiperpar\'ametros.
\par
Durante el proceso de etiquetado, se hall\'o que el modelo para
predecir lemas y etiquetas \textit{POS} mostr\'o un mejor rendimiento
en la predicci\'on de estas \'ultimas que en las primeras. No solo
porque fue necesario corregir manualmente una mayor cantidad de lemas
que de etiquetas \textit{POS}, sino porque, adem\'as, el porcentaje
de palabras en las cuales el modelo no acert\'o a predecir el lema
correcto en ninguna ocurrencia fue mayor al caso an\'alogo de etiquetas
\textit{POS}. 
\par
Por limitaciones de tiempo y recursos disponibles para el etiquetado,
se decidi\'o simplificar casos como verbos reflexivos y cl\'iticos. En
estas situaciones los verbos fueron transcriptos en infnitivos y las
marcas reflexivas y de cliticidad fueron omitidas.
Una posible mejora a este procedimiento consiste en repensar si esas
decisiones fueron las mejores y qu\'e impacto tienen sobre los datos
(cu\'antos verbos se ven afectados y en qu\'e medida). Del mismo modo,
podr\'ia ser de inter\'es contar con m\'as de una anotaci\'on que
permitiera validar de forma m\'as robusta la calidad de las etiquetas
obtenidas. Para esto, lo \'optimo ser\'ia contar con anotaciones
humanas que siguieran un protocolo estandarizado\footnote{Las gu\'ias
de anotaci\'on elaboradas en este trabajo persiguen un prop\'osito
semejante. Si bien aqu\'i las anotaciones fueron realizadas por una sola
persona, estas gu\'ias proporcionaron uns forma de fijar reglas que
pudieran ser recordadas en las distintas sesiones de anotaci\'on.}, dado
que esto permite un mejor control de los criterios de las 
etiquetas a utilizar y la fiabilidad del procedimiento. Sin embargo,
contar con anotaciones provenientes de distintos algoritmos de etiquetado
tambi\'en puede ser provechoso, y los resultados podr\'ian ser usados
para comparar estos algoritmos.
\par
Respecto de la vectorizaci\'on, la figura \ref{fig-results-features-fit-time}
muestra que los vectorizadores basados en \textit{ratio} de \textit{odds}
y en \textit{TF-IDF} son los que mayor tiempo de ajuste requirieron. No
obstante, el encontrarse todos los vectorizadores en un rango de tiempo
de ajuste muy similar y, adem\'as, casi despreciable, este dato no
constituy\'o una variable de peso a la hora de seleccionar el mejor vectorizador.
Puesto que fue la implementaci\'on basada en \textit{TF-IDF} con
logaritmo sobre la frecuencia inversa en los documentos la que
mejores resultados mostr\'o respecto de las m\'etricas de evaluaci\'on consideradas,
se opt\'o por esta vectorizador para el modelo final. Cabe destacar que, si bien
sus resultads respecto de la precisi\'on y la cobertura no fueron los \'optimos,
s\'i mostr\'o un mejor balance entre estas dos medidas, tanto en el promedio
general, como en el balance por clase (ver tabla \ref{table-results-vectorizers-val}).
\par
Para la evaluaci\'on de los vectorizadores, se fij\'o en $300$
el n\'umero de dimensiones a generar, que no es otra cosa que la
cantidad de palabras consideradas como relevantes para caracterizar
ambos grupos de discursos a predecir. Podr\'ia
ser se inter\'es contrastar si se observan los mismos resultados al
disminuir o incrementar el n\'umero de dimensiones. Asimismo, en la
implementaci\'on desarrollada, las t\'ecnicas estad\'isticas reseñadas por
\cite{monroe2008fightin} se emplean solamente
para la selecci\'on de palabras pero, una vez seleccionadas, todos
los vectorizadores siguen el mismo procedimiento: cuentan la frecuencia
absoluta de tales palabras en los documentos. En una futura iteraci\'on
sobre este trabajo, se podr\'ian implementar vectorizadores que
no solo utilizasen tales t\'ecnicas estad\'isticas en la selecci\'on
de palabras sino tambi\'en en la asignaci\'on de pesos a los vectores resultantes.
\par
Previo al entrenamiento del modelo de regresi\'on log\'istica, se
realiz\'o una validaci\'on cruzada para estimar los mejores hiperpar\'ametros
para el ajuste. Se encontr\'o que cualquier conjunto de hiperpar\'ametros
que no implicase regularizaci\'on ($penalty=None$) y utilizase
la proporci\'on de la clase objetivo en la estimaci\'on
($class\_weight=\lbrace0:0.56, 1:0.44\rbrace$). El par\'ametro que define
la intensidad inversa de la regularizaci\'on a aplicar es desestimado
puesto que no se aplica tal regularizaci\'on. El modelo final entrenado
mostr\'o un mejor rendimiento que aquellos pre-entrenados
durante la selecci\'on del vectorizador y los hiperpar\'ametros, resultado que
puede ser asociado al incremento en los datos durante el entrenamiento,
puesto que esta versi\'on fue ajustada utilizando todos los datos
de entrenamiento y no con el $75\percentsign$ de los mismos,
como se hizo previamente.
\par
La experimentaci\'on realizada en este trabajo conduce as\'i a elegir
la t\'ecnica basada en \textit{TF-IDF} con logaritmo sobre la
frecuencia inversa de los documentos como la mejor para
realizar una selecci\'on de rasgos que puedan ser utilizados al
vectorizar los documentos. En futuros trabajos ser\'ia de inter\'es
contrastar los resultados aqu\'i observados con los que pudieran obtenerse
al seguir un experimento similar en otro tipo de modelos.
La regresi\'on log\'istica constituye un modelo discriminativo, es decir
que aprende rasgos que le permiten diferenciar entre las clases
a predecir, aunque estos rasgos no sean necesariamente intr\'insecos
a las clases mismas. Los modelos generativos como Naïve Bayes, en cambio, estiman un
modelo posible que podr\'ia haber dado lugar a las clases observadas y
luego, ante una nueva observaci\'on, calculan las probabilidades de que
las distintas clases puedan ser generadas por esa observaci\'on.
Evaluar las t\'ecnicas aqu\'i estudiadas con este tipo de modelos podr\'ia
conducir a la obtenci\'on de resultados m\'as robustos.
\par
Por \'ultimo, cabe notar que los datos utilizados en este trabajo
no han sido voluminosos. Esto se debe a que, tras la descarga
de la versi\'on taquigr\'afica de la sesi\'on, fue necesario un
meticuloso procedimiento de limpieza y preprocesamiento que permitiese
hacer uso de esos datos. Incluir otras sesiones hubiese implicado
una labor mucho mayor y las limitaciones de tiempo y recursos no
lo permitieron. Sin embargo, ser\'ia deseable aumentar, en trabajos
venideros, los datos sobre los cuales se han desarrollados los procedimientos
hasta aqu\'i detallados, de modo que los resultados puedan verificarse en
\textit{corpora} que no solo sean mayores en tamaño sino que, adem\'as,
aborden otras tem\'aticas y, en ese sentido, presenten una mayor
diversidad. La mejora en los resultados observados en el entrenamiento
final constituyen un primer indicio del beneficio que tal aumento podr\'ia
significar.