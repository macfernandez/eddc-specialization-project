La sesión analizada contó con un total de 70 senadores,
pertenecientes a 26 partidos políticos, distribuidos del
siguiente modo (Figura \ref{fig-distrib-senators}):

\begin{figure}[h!]
\centering
\includegraphics[scale=0.5]{../visualizations/senators_distribution.png}
\caption{Distribución de senadores en partidos políticos y en provincias.}
\label{fig-distrib-senators}
\end{figure}

Respecto de los partidos, solo uno (Frente de todos) cuenta con 15
senadores; también uno solo (Alianza frente para la victoria) es
representado por 10 senadores; dor partidos (Alianza cambiemos y
Juntos por el cambio) cuentan con 7 senadores, y el resto de los
partidos tienen entre 3, 2 y un senador. Y en cuanto a las provincias
(24 en total), la mayoría tiene 3 senadores exceptuando a Tucumán y
a La Rioja, que tienen solo 2.\todo[color=pink]{Agregar tablas en anexo
dependiendo de la extensión del trabajo.}

La figura \ref{fig-distrib-senators} nos muestra además que la intención
de voto no guarda una relación unívoca con los partidos a los cuales los
senadores representan. A excepción de los partidos que cuentan con un único
senador, la mayoría cuenta con senadores que votaron a favor y senadores
que votaron en contra de la ley para el acceso al aborto.

\begin{figure}[h!]
\centering
\includegraphics[scale=0.5]{../visualizations/senators_vote.png}
\caption{Distribución votos en los distintos partidos políticos.}
\label{fig-distrib-vote}
\end{figure}

Por último, al considerar las intervenciones discursivas, observamos que
9 senadores no intervinieron en la sesión, y las intervenciones de los
restantes 61 senadores fueron resumidas en la Tabla \ref{table-tokens}\todo[color=pink]{revisar numeración de tabla}, que plasma
las métricas univariadas de la cantidad de tokens totales y únicos y nos
permite observar que ambas obedecen una distribución sesgada a izquierda.

\begin{figure}[h!]
\centering
\includegraphics[scale=0.6]{../visualizations/tokens_vote.png}
\caption{Distribución de tokens emitidos por los oradores en relación a
su intención de voto.}
\label{fig-distrib-tokens-vote}
\end{figure}

La Figura \ref{fig-distrib-tokens-vote}, por otro lado, nos muestra,
de forma más detallada, la distribución de los tokens emitidos según
la intención de voto de los oradores.

\begin{table}[h!]
\begin{center}
\begin{tabular}{ |c|c|c|c| }
\hline
Tokens & Media & Mediana & Desvío Estándar \\
\hline\hline
Totales & 811.65 & 636.12 & 714.79 \\
\hline
Únicos & 296.47 & 254.25 & 238.68 \\
\hline
\end{tabular}
\caption{Metrícas univariadas de la cantidad de tokens}
\label{table-tokens}
\end{center}
\end{table}
